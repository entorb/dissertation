% % Farbdefinitionen aus tudbeamer vorlage
% %\definecolor{tublau}{RGB}{0,29,75} % aus OpenOffice Vorlage -> sieht falsch aus, daher cmyk
% \definecolor{tublau}{cmyk}{0.86,0.48,0,0.68} % aus OpenOffice Vorlage

% aus TUD Farbregister
\definecolor{tublau100}{HTML}{0B2A51}
\definecolor{tublau90}{HTML}{1D355B}
\definecolor{tublau80}{HTML}{2F4067}
\definecolor{tublau70}{HTML}{4C7AB9}
\definecolor{tublau60}{HTML}{6185C0}
\definecolor{tublau50}{HTML}{7392C9}
\definecolor{tublau40}{HTML}{87A1D2}
\definecolor{tublau30}{HTML}{9CB1DB}
\definecolor{tublau20}{HTML}{B8C6E6}
\definecolor{tublau10}{HTML}{D8E0F2}
\definecolor{tuauszeich1}{HTML}{0059A3}%blau
\definecolor{tuauszeich180}{HTML}{346FB2}%blau 80%
\definecolor{tuauszeich140}{HTML}{87A1D2}%blau 80%
\definecolor{tuauszeich2}{HTML}{51297F}%dk lila
\definecolor{tuauszeich250}{HTML}{B07CAE}%dk lila
\definecolor{tuauszeich3}{HTML}{811A78}%magenta
\definecolor{tuauszeich4}{HTML}{007A47}%grün

\definecolor{tublau}{HTML}{0B2A51} % tublau100

% Eigene Latex Farben für Mat Namen
\definecolor{CrPd}{HTML}{FF98FF}
\definecolor{WPd}{HTML}{00FF00}
\definecolor{DMBI}{HTML}{FF9D00}
\definecolor{AOB}{HTML}{00FFFF}
% {0,1,0}

% 
% % Farbdefinitionen aus TUBeamervorlage
% \definecolor{tuwhite}{gray}{1.00}
% \definecolor{tublack}{gray}{0.00}
% \definecolor{tuskyblue}{rgb}{0.80, 0.80, 1.00}
% \definecolor{tublue}{rgb}{0.20, 0.20, 0.80}
% \definecolor{tudarkblue}{rgb}{0.04, 0.16, 0.32}
% % \definecolor{extradarkblue}{rgb}{0.00, 0.15, 0.36}
% \definecolor{tuextradarkblue}{cmyk}{1.00, 0.70, 0.10, 0.50}
% \definecolor{tuturquoise}{rgb}{0.00, 0.80, 0.60}
% \definecolor{tugray}{gray}{0.59}
% \definecolor{tudarkgray}{gray}{0.50}
% 

%
% \definecolor{HKS92K100}{cmyk}{0.1,0.00,0.05,0.65}
%
% \definecolor{HKS44K100}{cmyk/rgb}{1.00,0.50,0.0,0.0/0,0.34902,0.639216}
% \colorlet{alert}{HKS44K100}
%

\setbeamercolor{normal text}{fg=white,bg=tublau}

\setbeamercolor{note text}{fg=black,bg=white}

%
%
\setbeamercolor{structure}{fg=tublau50,bg=tublau}
%
\setbeamercolor{framefoot}{fg=gray,bg=tublau}
\setbeamercolor{framehead}{fg=white,bg=tublau}
% %
% %
% %
% % % meine Grafiken kommen in solch eine Box
\setbeamercolor{colorfigurebox}{fg=tublau,bg=white}

% \setbeamercolor*{block body title}{bg=red,fg=white}

\setbeamercolor{block title}{bg=tuauszeich1,fg=white}
\setbeamercolor{block body}{bg=tublau80,fg=white}

% \setbeamercolor{item projected}{bg=tuauszeich2,fg=white}

%  \setbeamercolor{section number projected}{bg=tuauszeich2,fg=white}
\setbeamerfont  {section number projected}{family=\sffamily,series=\mdseries,size=\normalsize}

% Analog block body, title, example etc.

% \setbeamercolor*{separation line}{fg=white}
% \setbeamercolor{section in head/foot}{fg=white,bg=tublau}
% \setbeamercolor{subsection in head/foot}{fg=white,bg=tublau}
% \setbeamercolor*{author in head/foot}{fg=white,bg=tublau}
% \setbeamercolor*{title in head/foot}{fg=white,bg=tublau}
% \setbeamercolor*{date in head/foot}{fg=white,bg=tublau}
% \setbeamercolor*{titlelike}{fg=white,bg=tublau}

% \setbeamercolor*{palette primary}{fg=white,bg=tublau}
% \setbeamercolor*{palette secondary}{fg=white,bg=tublau}
% \setbeamercolor*{palette tertiary}{fg=white,bg=tublau}
% \setbeamercolor*{palette quaternary}{fg=white,bg=tublau}