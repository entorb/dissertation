\subsection[Seebeck]{Seebeckeffekt}

\begin{frame}{Seebeckeffekt}%
\spalten[4.5cm]{
~\bildbox{pics/ThomasSeebeck.jpg}{4cm}~%
}{
Thomas Johann Seebeck, 1821:
\par\vspace*{3cm}

Temperaturgradient \\ entlang Metall oder Halbleiter \\ erzeugt elektrische Spannung
}
\begin{textblock}{31}[1.0,0.0](105,31)
\includegraphics[page=1]{pics/skizze-dT-erzeugt-V.pdf}
\end{textblock}
\begin{textblock}{16}[0.0,0.0](10,78)
\tiny Bild: Wikipedia
\end{textblock}

\end{frame}

\begin{frame}{Seebeckeffekt in konv. HL}%
\begin{itemize}
% \item Seebeckeffekt: Temp. Unterschied \\ entlang Metall oder HL. \\ erzeugt Spannung% \pause
\item <1-> Seebeck Koeffizient $\S=\frac{V_\text{th}}{T_2-T_1}$
\item <1-> Vorzeichen \pfeil Majoritätsladungsträger % von \S \\ 
% \pause
% \pause
% \item \S $\propto$ (Fermi-Niveau \Ef\, --\,\, Transport-Niveau \Et) \quad $\S = \frac{\Ef-\Et}{e T}$
\item <2-> \S $\propto$ Diff. Fermi (\Ef) und Transport (\Et) Niveau %$\S \approx \frac{\Ef-\Et}{e \Tm}  \fnSym$
%\\ \hspace*{3cm} $\S \approx \frac{\Ef-\Et}{e T}  \fnSym$ %
% approx, weil konst A weggelassen
% \only<2->\zitat{Schmechel, JAP 93, 4653 (2003)}
\end{itemize}
\begin{textblock}{31}[1.0,0.0](124,14)
\includegraphics[page=1]{pics/skizze-dT-erzeugt-V.pdf}
\end{textblock}
\only<2->{
\vspace*{1ex}
\begin{center}
\includegraphics[page=1]{pics/skizze-Es-AHL.pdf}
\end{center}
}
% \only<1->{%
% \begin{textblock}{20}[0.0,0.0](98,25)
% $\S=\frac{V_{12}}{T_2-T_1}$
% \end{textblock}
% }
\only<2->{%
\fn{Fritzsche, SSC 9, 1813 (1971)}
\begin{textblock}{25}[0.0,0.0](104,26)% 26 {3} (0,0) % pagewidth=128mm, 15mm border to page top, [1.0,0.0] = rel. box coordinates
$\S \approx \frac{\Ef-\Et}{e \Tm}  \fnSym$
% \\
% $\Es := \Ef-\Et$
\end{textblock}
}
\only<2->{%
\begin{textblock}{20}[0.0,0.0](104,54)
$\S \approx \frac{\,\Es\,}{e \Tm}$
\end{textblock}
}
%
% \todo{}
\ton{pos/neg \S\&\Es, \Tm, T \pfeil Drift \pfeil Diff}
\end{frame}

\begin{frame}{Seebeckeffekt in org. HL}%
% \vspace*{1ex}
Für organische Halbleiter mit gaußförmiger Zustandsdichte lässt sich ein \Et definieren\zitat{Schmechel, JAP 93, 4653 (2003)}
\vspace*{4ex}
\begin{equation*}
\Et :=  \frac
{ \int_{-\infty} ^{+\infty} E ~ \c'(E) ~ dE }
{ \int_{-\infty} ^{+\infty} \c'(E) ~ dE } 
= \frac{1}{\c} \int_{-\infty} ^{+\infty} E  ~ \c'(E) ~ dE
\end{equation*}
\\\vspace*{2.5ex}
für das gilt
\vspace*{3ex}
\begin{equation*}
\S = \frac{\Ef-\Et}{e T} = \frac{\,\Es\,}{e T} % \quad \text{mit} \quad \Es := \Ef-\Et
% \\ &= 
\end{equation*}
\end{frame}

\begin{frame}{Freie Ladungsträger}%
\vspace*{-2ex}
\begin{center}
Dichte freier Ladungsträger (n-dotiert)
\end{center}
\vspace*{-1ex}
\begin{equation*}
\n = \int_{-\infty}^{\infty} \textcolor{WPd}{\fFD(E,\Ef,T)} \cdot \textcolor{CrPd}{\dos(E)} dE
\end{equation*}
\vspace*{-3ex}
\spalten{}{%
\only<2>{%
\begin{block}
{\centering\phant Konventionelle HL \phant}
\vspace*{-1ex}
\begin{align*}
\dos(E) &\propto \sqrt {E-\Ec} \\
\Pfeil \n &\propto \exp\left(-\frac{E_\text{C}-\Ef}{\kT}\right) \fnSym{}
\end{align*}
\vspace*{-0.75ex}
per Definition
\vspace*{-1ex}\begin{equation*}
E_\text{C}-\Ef =: |\Es|
\end{equation*}
\begin{equation*}
\textcolor{yellow}{|\Es| \downarrow \, \Leftrightarrow \n \uparrow}
\end{equation*}
\end{block}
}
\only<3>{%
\begin{block}
{\centering\phant Organische HL \phant}
\par
\vspace*{2ex}
% \begin{itemize}
- exakte \dos unbekannt
\\- Gauß-Verteilung üblich angenommen
{\small $\Pfeil \n \propto \exp{\left(-\frac{ |\Es|+|\Et| - \frac{\gausswidth^2}{2\kT}}{\kT} \right)} \fnSym{}$ }
\\gleicher Trend
\begin{equation*}
\textcolor{yellow}{|\Es| \downarrow \, \Leftrightarrow \n \uparrow}
\end{equation*}
\end{block}
}%
}

\begin{textblock*}{50mm}[0,1](8mm,85.5mm) % Seite: 128mm x 96mm
% \includegraphics[width=4cm]{plot/sim_Fermi-DOS-Es-org.pdf} %sim_Fermi-DOS-Es-org
\only<2>{\bildbox[scaled]{plot/sim_Fermi-DOS-Es-csc.pdf}{50mm}}
\only<3>{\bildbox[scaled]{plot/sim_Fermi-DOS-Es-org.pdf}{50mm}}
\end{textblock*}
%
\ton{Dot schiebt \Ef!}
\only<2->{
\begin{textblock}{64}[0.0,0.0](66,85)
\tiny{\fnSym{}Boltzmann Näherung verwendet}%
\end{textblock}%
}

\end{frame}

