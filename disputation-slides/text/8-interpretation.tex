\subsection[Leitfähigkeit]{Leitfähigkeitsdaten}
\begin{frame}{Ansatz}
\spalten[0.6\textwidth]{
\begin{block}{}
\vspace*{-3ex}
\begin{align*}
\onslide<1->{\c &= e \cdot \mob \cdot \n \\}
\onslide<2->{\n &= \DopEff \cdot \nD \\}
\onslide<3->{\n &= \DopEff \cdot \nM \cdot \frac{\C}{1+\C} \\}
\onslide<4->{\Pfeil \textcolor{yellow}{\c} &= e \cdot \textcolor{yellow}{\mu} \cdot \textcolor{yellow}{\DopEff} \cdot \nM \cdot \frac{\C}{1+\C}}
% \\ \Pfeil \c &\propto \mu \cdot \DopEff
\end{align*} 
\end{block}
}{
}
%
\begin{textblock*}{20ex}[1,0](118mm,25mm) % Seite: 128mm x 96mm
\scriptsize
%\vspace*{2cm}
\begin{tabular}{r@{ : }l}
$e$ & Elementarladung\\
\mob & Beweglichkeit\\
\n & Ladungsträgerdichte\\
\DopEff & Dotiereffizienz\\
\nD & Dotandendichte\\
\nM & Moleküldichte\\
\C & Dotierkonzentration (MR)\\
\end{tabular} 
\end{textblock*}
%
\ton{Annahme: Dotand Mol ersetzt Host, \mob oder \DopEff festhalten und über anderen lernen}
\end{frame}

% \subsection[\mobLL]{Untergrenze Beweglichkeit}
\begin{frame}{Untergrenze Beweglichkeit}
\spalten[0.43\textwidth]{
\begin{block}{}
\small
\vspace*{-3ex}
\begin{align*}
% $\c = e \cdot \mu \cdot \DopEff \cdot \nM \cdot \frac{\C}{1+\C}$\par
% \vspace*{.5cm}
\DopEff[]{}_\text{,max} = \SI{100}{\percent} %\hspace*{4ex}
\phantom{\cmVs{5} \fnSym{}}
% phantom, damit genauso hoch wie auf nächster Seite
% \par\vspace*{.5cm}
\\
% \Pfeil 
\mobLL = \frac{\c}{e \cdot 100\,\percent \cdot \nM \cdot \frac{\C}{1+\C}}
\end{align*}
\end{block}
\only<2->{
\begin{itemize}
\item $\mobLL \approx 0.01 \dots \cmVs{1}$\\~
\item Für \C\pfeil 0 muss aber
$\mob \pfeil \mob(\text{C}_{60})$ gelten
\end{itemize}
}
}{
\only<2->{\bildbox{plot/calc-mobLL-n.pdf}{6cm}}
}
\ton{Graphen nicht beschreiben, GrOrdnung reicht}
\end{frame}

\begin{frame}{Untergrenze Dotiereffizienz}
\spalten[0.43\textwidth]{
\begin{block}{}
\small
\vspace*{-3ex}
\begin{align*}
\mobUL(\text{C}_{60}) = \cmVs{5} \fnSym{}
\hspace*{4ex}
% \Pfeil
\\
\DopEffLL = \frac{\c}{e \cdot \mobUL \cdot \nM \cdot \frac{\C}{1+\C}}
\end{align*}
\end{block}
\vspace*{-1ex}
% \only<2->{
\begin{itemize}
\item <2-> \CrPd: \DopEff $>20\percent$
\item <2-> \dmbi: \DopEffLL $\uparrow$
\only <3>  {\item \Pd:\\ \DopEffLL $\downarrow$}
\only <4-> {
\item \Pd:\\ \DopEffLL $\downarrow$ \& \n{}$_\text{,min}$ $\downarrow$
\\im Widerspruch zu Seebeckdaten
\\\Pfeil Beweglichkeit $\downarrow$
}
\end{itemize}
\fn{OFET by Itaka\etal AM 18, 1713 (2006)}
% }

}{
\only<2-3>{\bildbox{plot/calc-nLL-n.pdf}{6cm}}
\only<4->{\bildbox{plot/calc-nLL+effLL-n.pdf}{6cm}}
% \only<4->{
% \begin{textblock}{50}[0.0,0.0](17,78)% mm
% \footnotesize \nM{}(\CS) $= 1.36\cdot10^{21}\,\text{cm}^{-3}$
% \end{textblock}
% }
% \only<2->{
% \begin{textblock}{20}[1.0,0.0](118,30)% mm
% \includegraphics{pics/legende.pdf}
% \end{textblock}
% }
}
\todo{cite Mityashin\etal AM 24, 1535 (2012)}
\ton{CrPd 1:5, DMBI Anstieg wohl echt, Da S monoton, wenn mob @ kl \C gleich \pfeil konst offset}
\end{frame}

\subsection[Seebeck]{Seebeckdaten}
\begin{frame}{Ansatz}
\vspace*{-3ex}
\begin{equation}
 \n = \int^\infty_{-\infty} \textcolor{WPd}{\fFD(E,\textcolor{yellow}{\Ef})} \cdot \textcolor{CrPd}{\dos(E)} dE \overset{!}{=} \DopEff\cdot\nD(\C) 
\end{equation}
\only<2->{
\\\vspace*{4ex}
\Pfeil Bestimmung von $\textcolor{yellow}{\Ef}(\DopEff,\C)$
\\durch iterative numerische \\Integration bis (1) erfüllt ist
%
\begin{textblock*}{26ex}[0,1](10mm,86mm)
\begin{block}{}
\vspace*{-0.5ex}
% \begin{align*}
$\textcolor{WPd}{\fFD(E,\textcolor{yellow}{\Ef})} = \frac{1}{1+\exp{\left(\frac{E-\textcolor{yellow}{\Ef}}{\kT}\right)}}$
\\\vspace*{1ex}{\small Annahme: gaußförmige $\dos(E)$}
\\\vspace*{1ex}$\textcolor{CrPd}{\dos(E)}= \frac{\nH(\C)}{\sqrt{2 \pi} ~ \gausswidth} \exp{\left(-\frac{E^2}{2\gausswidth^2}\right)}$
% \end{align*}
\end{block}
\end{textblock*}
}
\only<2->{
\begin{textblock*}{20ex}[1,0](118mm,60mm) % Seite: 128mm x 96mm
\scriptsize
\begin{tabular}{r@{ : }l}
\n & Ladungsträgerdichte\\
\fFD & Fermi-Dirac Distribution\\
\dos  & Zustandsdichte \\
\DopEff & Dotiereffizienz\\
\C & Dotierkonzentration (MR)\\
\nD & Dotandendichte\\
\nH & Wirtdichte\\
\gausswidth & Breite d. gaußf. $\dos$\\
% \mob & Beweglichkeit\\
% \nM & Moleküldichte\\
\end{tabular}
% }
\end{textblock*}
% 
\begin{textblock*}{4cm}[1,1](113mm,57mm) % Seite: 128mm x 96mm
% \includegraphics[width=4cm]{plot/sim_Fermi-DOS-Es-org.pdf} %sim_Fermi-DOS-Es-org
\bildbox[scaled]{plot/sim_Fermi-DOS-Es-org.pdf}{30mm}
\end{textblock*}
}
% 
\end{frame}

\begin{frame}{Berechnung des Fermi-Niveaus} %\Ef (\C)
\begin{center}
~\bildbox{plot/calc-Ef-von-DopEff.pdf}{67mm}
\\~für \gausswidth[100]\fnSym
\end{center}
\fn{OFET by Fishchuk\etal, PRB 81, 045202 (2010): $\gausswidth[](\text{C}_{60}) = \meV{88}$}
\only<2->{
\begin{textblock*}{13mm}[1,0.5](118mm,43mm) % Seite: 128mm x 96mm
\bildbox{pics/skizze-Energien.pdf}{12.6mm}
\end{textblock*}
}
% \ton{0 = Gaussmax}
\end{frame}

\begin{frame}{Abschätzung des Transport-Niveaus}
% \centering
\begin{center}
\vspace*{-2ex}
~Einbeziehung der Seebeck Daten: $\Ef + |\Es| = \Et$
% \hspace*{2mm}
\only<1>{~\bildbox[unscaled]{plot/calc-ETr-von-Es+DopEff.pdf}{65mm}}
\only<2>{~\bildbox[unscaled]{plot/calc-ETr-von-Es+DopEff+linie.pdf}{65mm}}
\end{center}
\ton{\gausswidth[100]}
\end{frame}

\begin{frame}{Annahme konst. Transport-Niveau}
\spalten[5.75cm]{
Annahmen konstanter
\begin{itemize}
\item $\Et[-225]$
\item $\gausswidth[100]$
\item $\nM{} = 1.36\cdot10^{21}\,\text{cm}^{-3}$
\end{itemize}
% \\- konst. $\Et[-225]$
% \\- konst. $\gausswidth[100]$
% \\- konst. $\nM{} = 1.36\cdot10^{21}\,\text{cm}^{-3}$
% \\\vspace*{1ex}
\par
\vspace*{2ex}ermöglichen Berechnung von 
\begin{itemize}
\item Ladungsträgerdichte $\n(\S)$
\item Dotiereffizienz $\DopEff(\S)$
\item Beweglichkeit $\mob(\S,\c)$
\end{itemize}
% 
% \\- Ladungsträgerdichte $\n(\S)$
% \\- Dotiereffizienz $\DopEff(\S)$
% \\- Beweglichkeit $\mob(\S,\c)$
}{
\only<2->{
\bildbox[unscaled]{plot/calc-constEtr-n-225.pdf}{5.0cm}
}
}
% \hspace*{0mm}% wichtig weil sonst bildbox nicht mittig
% \bildbox[scaled]{plot/calc-ETr-von-Es+DopEff.pdf}{100mm}
% \end{center}
% \ton{\gausswidth[100]}
%
\only<2->{
\begin{textblock*}{19mm}[1,1](126mm,38mm) % Seite: 128mm x 96mm
\includegraphics{pics/legende.pdf}%
\end{textblock*}
% \begin{textblock}{50}[0.0,0.0](17,78)% mm
% \footnotesize \nM{}(\CS) $= 1.36\cdot10^{21}\,\text{cm}^{-3}$
% \end{textblock}
}
%
\todo{Text: Bilder beschreiben, Bilder einzeln?}
\ton{\n Sät.@ E19, \DopEff $\downarrow$, \mob $\downarrow$ f. Pd}
\end{frame}
