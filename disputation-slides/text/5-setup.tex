\subsection{Messaufbau}

\begin{frame}{Probenpräparation}
\spalten{%
\centering%
~\bildbox{pics/setup-photo-nur-kammer.jpg}{5cm}~%
}{%
\centering%
~\bildbox{pics/evap.pdf}{5cm}~%
}
\ton{Präparation u Mess insitu, 0.8nm/min}
\end{frame}

\begin{frame}{Messmodi}%
\spalten[6cm]{ %5.8
% Kontaktierung \\
%\\anlegen und messen\\von $V$ und $I$
% \\\vspace*{2ex}
% Messmodi:
\textbf{1. Leitfähigkeit \boldmath \c}
\\$V$ anlegen (\V{1})
\\$I$ messen \pfeil \c
\\\vspace*{2ex}
\textbf{2. Seebeck Koeff. \boldmath \S}
\\$\Delta\T$ anlegen (\K{5}) % =\T_2-\T_1
\\$V_\text{th}$ messen \pfeil \S
\\\vspace*{4ex}

{
\small Variation von
\vspace*{1ex}
\begin{itemize}
\item Material
\item Dotierkonzentration \C
\item (Temperatur \T)
\end{itemize}
}
}{
}
\begin{textblock}{76}[1,0](123,16)
\includegraphics{pics/Setup-ani-noAni.pdf}
\end{textblock}
\begin{textblock}{40}[1,1](117,78)
% \includegraphics[width=30mm]{pics/chiffre.pdf}
\tiny
Chiffre 0815:\\
Alleinstehende Vakuumkammer sucht \\
Begleiter für einsame Stunden
\end{textblock}
%
\ton{30nm = 1000stel Haar}
\end{frame}

