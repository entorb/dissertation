\subsection[Leitfähigkeit]{Leitfähigkeitsstudie}%

\begin{frame}{Leitfähigkeit \vs Dotierung}%
\only<1-2>{\AS Dotanden}%
\only<3-4>{\Pd Dotanden}%
\spalten[0.55\textwidth]% leer wichtig?
{%
\only<1>{\bildbox[unscaled]{plot/MR-Cond-n-AS-evap.pdf}{6cm}
 \par\vspace*{0.3ex}\tiny{direkt nach Präparation, \Tmess[25] \phant }}%
\only<2>{\bildbox[unscaled]{plot/MR-Cond-n-AS-evap+25.pdf}{6cm}
 \par\vspace*{0.3ex}\tiny{nach Ausheizen 1\,h bei \T[100], \Tmess[25] \phant }}%
\only<3>{\bildbox[unscaled]{plot/MR-Cond-n-Pd-evap.pdf}{6cm}
 \par\vspace*{0.3ex}\tiny{direkt nach Präparation, \Tmess[25] \phant }}%
\only<4>{\bildbox[unscaled]{plot/MR-Cond-n-Pd-evap+30.pdf}{6cm}
 \par\vspace*{0.3ex}\tiny{nach Ausheizen 1\,h bei \T[70], \Tmess[30] \phant }}%
\par
%\only<5>{~\bildboxAlt{plot/MR-Cond-n-All-evap.pdf}{6cm}~\\after thermal annealing}%
%\end{center}%
}%
{
\begin{itemize}
\item $\C \uparrow ~ \Pfeil \, \c\uparrow$
\only <1> {\item \dmbi $\approx 100\times$ höhere \c als \aob}
\only <2> {
\item Ausheizen: starke Verbesserung für \aob
\item Vermutlich Umformung der Vorstufe nicht abgeschlossen
\item Maximum:\\
\Scm{5.3} @ \C[0.650]
}

\item <3-> Abnahme bei hohen \C
\item <4-> Ausheizen: \\Zunahme bei kl. \C \\Abnahme bei gr. \C
\item <4-> Maximum:\\\Scm{4.3} @ \C[0.045] % 4.3 steht in T=30 Datei!!!
\only<4->{\ton{Zuname=Ausrichten, Abnahme=gr Pd Mol stören}}
\only<4->{
\begin{textblock}{30}[1,1](113,82)% mm
\bildbox[scaled]{pics/mat-C60+Pd.jpg}{3cm}
\end{textblock}
}
\end{itemize}
}
\begin{textblock}{50}[0,0.5](10,70)% mm
% \scriptsize
\footnotesize
% \begin{equation*}
$\text{\C (MR)} = \frac{\text{\# Dotand Mol.}}{\text{\# Wirt Mol.}}$
% \end{equation*}
\end{textblock}
\begin{textblock}{50}[0,0.5](10,78)% mm
\footnotesize
\CS undotiert: $\c\approx \Scm{e-8}$ \fnSym
\\Rekordwert: $\c\approx \Scm{10} ^{\ddagger}$
\fn{Li \etal JAP 100, 23716 (2006) ; $^\ddagger$Olthof\etal PRL 109, 176601 (2012)}
\end{textblock}
%
\only<1>{\ton{DopLogSkala}}
\end{frame}

\begin{frame}{Leitfähigkeit \vs Dotierung}%
alle 4 Dotanden%
\spalten[.55\textwidth]%
{%
\only<1>{\bildbox[unscaled]{plot/MR-Cond-n-AS25+Pd30.pdf}{6cm}}%
\only<2>{\bildbox[unscaled]{plot/MR-Cond-n-AS25+Pd30+fit.pdf}{6cm}}%
\par\vspace*{0.3ex}\tiny{nach Ausheizen, $T=25$ / \grad{30}}%
}%
{%
\begin{itemize}
%\item <1-> low doping: reactive dopants better than air-stable ones
% \item Nahe an Literaturrekordwert \Scm{10}
\item <2-> $\c(\C)$ meist linear
\\superlinear für \dmbi
\item <2-> \Pfeil Dotiereffizienz $\uparrow$ oder Beweglichkeit $\uparrow$
\end{itemize}

}
% \ton{}
\end{frame}

\subsection[Seebeck]{Seebeckstudie}
% \subsection[\S \vs \C]{Seebeck \vs Dotierung}

\begin{frame}{Seebeck \vs Dotierung}%
\spalten[0.55\textwidth]{%
\only<1>{\bildbox[unscaled]{plot/MR-See+Es-n1}{6cm}}%    AOB
\only<2>{\bildbox[unscaled]{plot/MR-See+Es-n12}{6cm}}%   DMBI
%\only<4>{\bildbox[unscaled]{plot/MR-See+Es-n123}{6cm}}%  W2
\only<3>{\bildbox[unscaled]{plot/MR-See+Es-n1234}{6cm}}% Cr2
\scriptsize{$\Tm=\grad{40}$, nach Ausheizen}
}{%
\begin{itemize}
 \item <1-> $S$ neg. \pfeil e-Leitung
 \item <1-> Fermi-Niveau schiebt \pfeil Transport-Niveau
 \item <2-> \dmbi:\\\n gr. als für \aob
 \item <3-> \Pd Dotanden:\\ \n gr. als für \AS
 \item <3-> \n sinkt monoton
%  Wort Sättigung zu dreist  \item <3-> Sättigung um $40-\meV{50}$ für alle Dotanden
\end{itemize}
}
\ton{\n steigt nicht wieder an (vgl \c (Pd)}
\end{frame}

