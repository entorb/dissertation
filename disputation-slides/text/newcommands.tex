% geo: 128mm x 96mm

\newcommand{\grad}[1] {\SI{#1}{\celsius}}
\newcommand{\K}[1] {\SI{#1}{\kelvin}}
\newcommand{\Scm} [1] {\SI[per=frac,fraction=nice,valuesep=thick]{#1}{\siemens\per\centi\meter}}
\newcommand{\mV}  [1] {\SI{#1}{\milli\volt}}
\newcommand{\V}  [1] {\SI{#1}{\volt}}
\newcommand{\meV} [1] {\SI{#1}{\milli\electronvolt}}
\newcommand{\nm} [1] {\SI{#1}{\nano\meter}}
\newcommand{\mm} [1] {\SI{#1}{\milli\meter}}
\newcommand{\uVK} [1] {\SI{#1}{\mu\volt\per\kelvin}} % \micro -> Font eurm10 at 10pt not found
\newcommand{\cmVs}  [1] {\SI{#1}{\centi\meter\squared\per\volt\per\second}}
\newcommand{\mr}[1] {\SI{#1}{MR}}

\newcommand{\vs}[0] {vs.\ }
\newcommand{\etal}[0] { \mbox{et al.}\xspace}
\newcommand{\insitu}[0] {in~situ\xspace}

\newcommand{\phant}[0]{\hspace*{-2ex}\phantom{fg}} % invisible box for fixing jumping text

\newcommand{\pfeil} [0] {\ensuremath{\rightarrow}\xspace}
\newcommand{\Pfeil} [0] {\ensuremath{\Rightarrow}\xspace}

% \renewcommand{\thefootnote}{\fnsymbol{footnote}}

% \newcommand{\zitat} [1] {\footnote{\tiny{#1}}}
\newcommand{\zitat} [1] {\fnSym\fn{#1}}
\newcommand{\fnSym}[0]{\ensuremath{^\ast}} % \ast is schlecht weil in Gleichung verwendet
\newcommand{\fn} [1] {%
\begin{textblock}{108}[0.0,0.0](10,85)
\tiny{\fnSym#1}%
\end{textblock}%
}

% \newcommand{\cond} [0] {\ensuremath{\sigma}\xspace}
\let\corg\c
\renewcommand{\c} [1][\empty]%
{%
\ifthenelse{\equal{#1}{\empty}}%
{\ensuremath{\sigma}\xspace}%
{\mbox{\ensuremath{\sigma=\Scm{#1}}}}%
}
\newcommand{\C} [1][\empty]%
{%
\ifthenelse{\equal{#1}{\empty}}%
{\ensuremath{C}\xspace}%
{\mbox{\ensuremath{C=#1}}}%
}

\newcommand{\Eact} [1][\empty]{\ifthenelse{\equal{#1}{\empty}}%
 {\texorpdfstring{\ensuremath{E_\text{act,\c}}}{Eact}\xspace}%
 {\mbox{\ensuremath{E_\text{act,\c}=\meV{#1}}}}%
}
% \newcommand{\Eact} [0] {\texorpdfstring{\ensuremath{E_\text{act,\c}}}{Eact}\xspace}
\newcommand{\Es} [1][\empty]{\ifthenelse{\equal{#1}{\empty}}%
 {\ensuremath{E_\text{S}}\xspace}%
 {\mbox{\ensuremath{E_\text{S}=\meV{#1}}}}%
}
\newcommand{\Et} [1][\empty]{\ifthenelse{\equal{#1}{\empty}}%
 {\ensuremath{E_\text{Tr}}\xspace}%
 {\mbox{\ensuremath{E_\text{Tr}=\meV{#1}}}}%
}
\newcommand{\Ef} [0] {\ensuremath{E_\text{F}}\xspace}
\newcommand{\Ec} [0] {\ensuremath{E_\text{C}}\xspace}
\newcommand{\Ev} [0] {\ensuremath{E_\text{V}}\xspace}
\newcommand{\Egap} [0] {\ensuremath{E_\text{gap}}\xspace}
\newcommand{\kB} [0] {\ensuremath{k_\text{B}}\xspace}
\newcommand{\kT} [0] {\ensuremath{\kB T}\xspace}
\newcommand{\Tm} [0] {\ensuremath{T_\text{m}}\xspace}
\newcommand{\Tmess} [1][\empty]{\ifthenelse{\equal{#1}{\empty}}%
 {\ensuremath{T_\text{Mess}}\xspace}%
 {\mbox{\ensuremath{T_\text{Mess}=\grad{#1}}}}%
}
\newcommand{\T} [1][\empty]{\ifthenelse{\equal{#1}{\empty}}%
 {\ensuremath{T}\xspace}%
 {\mbox{\ensuremath{T=\grad{#1}}}}%
}
\let\Sorg\S
\renewcommand{\S} [0] {\ensuremath{S}\xspace}

\newcommand{\dos}     [0] {\ensuremath{D}\xspace}

\newcommand{\DopEff} [1][\empty]{\ifthenelse{\equal{#1}{\empty}}%
 {\ensuremath{\eta_\text{dot}}\xspace}%
 {\mbox{\ensuremath{\eta_\text{dot}=\SI{#1}{\percent}}}}%
}
\newcommand{\DopEffLL} [1][\empty]{\ifthenelse{\equal{#1}{\empty}}%
 {\ensuremath{\eta_\text{dot,min}}\xspace}%
 {\mbox{\ensuremath{\eta_\text{dot,min}=\SI{#1}{\percent}}}}%
}
\newcommand{\mob}  [1][\empty]{\ifthenelse{\equal{#1}{\empty}}%
 {\ensuremath{\mu}\xspace}%
 {\mbox{\ensuremath{\mu=\cmVs{#1}}}}%
}
\newcommand{\mobLL}  [1][\empty]{\ifthenelse{\equal{#1}{\empty}}%
 {\ensuremath{\mu_\text{min}}\xspace}%
 {\mbox{\ensuremath{\mu_\text{min}=\cmVs{#1}}}}%
}
\newcommand{\mobUL}  [1][\empty]{\ifthenelse{\equal{#1}{\empty}}%
 {\ensuremath{\mu_\text{max}}\xspace}%
 {\mbox{\ensuremath{\mu_\text{max}=\cmVs{#1}}}}%
}
\newcommand{\n} [0] {\ensuremath{n_\text{e}}\xspace}
\newcommand{\nLL} [0] {\ensuremath{n_\text{min}}\xspace}

\newcommand{\nM} [0] {\ensuremath{n_\text{Mol}}\xspace}
\newcommand{\nH} [0] {\ensuremath{n_\text{W}}\xspace}
\newcommand{\nD} [0] {\ensuremath{n_\text{D}}\xspace}

\newcommand{\fFD}     [0] {\ensuremath{f_\text{FD}}\xspace}
\newcommand{\gausswidth} [1][\empty]{\ifthenelse{\equal{#1}{\empty}}%
 {\ensuremath{\sigma_\text{G}}\xspace}%
 {\mbox{\ensuremath{\sigma_\text{G}=\meV{#1}}}}%
}

\newcommand{\gausscenter} [0] {\ensuremath{E_\text{G}}\xspace}

\newcommand{\meo} [0] {MeO-TPD\xspace}
\newcommand{\lili} [0] {BF-DPB\xspace}
\newcommand{\CSF} [0] {\texorpdfstring{C$_{60}$F$_{36}$}{C60F36}\xspace}
\newcommand{\FS} [0] {\texorpdfstring{F$_{6}$-TCNNQ}{F6TCNNQ}\xspace}

\newcommand{\AS} [0] {\textcolor{AOB}{luft}\textcolor{DMBI}{stabile}\xspace}
\newcommand{\Pd} [0] {\textcolor{CrPd}{luft}\textcolor{WPd}{reaktive}\xspace}
\newcommand{\CS} [0] {\texorpdfstring{C$_{60}$}{C60}\xspace}
\newcommand{\aob} [0] {\textcolor{AOB}{AOB}\xspace}
\newcommand{\CrPd}[0] {\textcolor{CrPd}{Cr$_2$(hpp)$_4$}\xspace} %Magenta
\newcommand{\WPd} [0] {\textcolor{WPd}{W$_2$(hpp)$_4$}\xspace}
\newcommand{\aobLong} [0] {3,6-bis(dimethylamino)acridine\xspace}
\newcommand{\dmbiPOHLong}[0] {2-(1,3-dimethyl-1\textsl{H}-benzoimidazol-3-ium-2-yl)phenolatehydrate\xspace}
\newcommand{\CrPdLong}[0] {tetrakis(1,3,4,6,7,8-hexahydro-2H-pyrimido[1,2-a]pyrimidinato)dichromium (II)\xspace}
\newcommand{\WPdLong}[0]  {tetrakis(1,3,4,6,7,8-hexahydro-2H-pyrimido[1,2-a]pyrimidinato)ditungsten (II)\xspace}
\newcommand{\dmbiPOH}[0] {\textcolor{DMBI}{DMBI-POH}\xspace}
\newcommand{\dmbi}[0] {\dmbiPOH}

% redefine the section
% \let\sectionorg\section % damit keine endlosschleife erzeugt wird
% \renewcommand{\section}[1]{\sectionorg{#1}\einzeiler{\huge\bfseries\thesection. #1}}
% \let\subsectionorg\subsection % damit keine endlosschleife erzeugt wird
% \renewcommand{\subsection}[1]{\subsectionorg{#1}\einzeiler{\large\bfseries\thesection.\thesubsection{} #1}}

% [1]=width colum1
% {2}=column1, {3}=column2
\newdimen\SpalteL%
\newdimen\SpalteR%

\newcommand{\spalten}[3][0.5\textwidth]{%
\SpalteL=0mm%
\SpalteR=0mm%
\advance\SpalteL by #1 %
\advance\SpalteR by \textwidth %
\advance\SpalteR by -\SpalteL %
\begin{columns}[T]%t/c/T alignment
\begin{column}{\SpalteL}%
#2%
\end{column}%
\begin{column}{\SpalteR}%
#3%
\end{column}%
\end{columns}%
}

\newcommand{\spaltenC}[3][\empty]{% Zentriert
\SpalteL=0mm%
\SpalteR=0mm%
\ifthenelse{\equal{#1}{\empty}}%
{\advance\SpalteL by 0.5\textwidth}
{\advance\SpalteL by #1}%
\advance\SpalteR by \textwidth%
\advance\SpalteR by -\SpalteL%
\begin{columns}[T]%t/c/T alignment
\begin{column}{\SpalteL}%
\centering%
#2%
\end{column}%
\begin{column}{\SpalteR}%
\centering%
#3%
\end{column}%
\end{columns}%
}

% \einzeiler[title]{text}
\newcommand{\einzeiler}[2][\empty]{%
\begin{frame}[c]
\begin{center}
\frametitle{#1\phant} \phant#2\phant
\end{center}
\end{frame}
}

%\todoframe[title]{Slope of Cond}
\newcommand{\todoframe}[1]{% [2][TODO]
\einzeiler{\textcolor{red}{\bfseries \Large TODO: #1}}
}

\newcommand{\todo}[1]{%
\begin{textblock}{128}[1.0,0.0](128,0)%
\tiny{\phant\textcolor{yellow}{\hfill TODO: #1}}%
\end{textblock}
}

\newcommand{\ton}[1]{%
\begin{textblock}{128}[0.0,1.0](0,95.5)%
\tiny{\phant\textcolor{yellow}{Ton: #1}}%
\end{textblock}
}

\newcommand{\bildbox}[3][scaled]{% [1]=unscaled  2=file; 3=width
% Achtung: rounded=true -> box + 4pt in jeder Richtung
\begin{beamercolorbox}[center,rounded=true,shadow=false,wd=#3]{colorfigurebox}%
\centering%
\ifthenelse{\equal{#1}{unscaled}}%
{\includegraphics[page=1]{#2}}%
{\includegraphics[page=1,width=#3]{#2}}%
\end{beamercolorbox}%
}

\newcommand{\bildframe}[4][scaled]{% [1]=unscaled  2=title 2=file; 3=width
\begin{frame}{#2}%
\bildbox[#1]{#3}{#4}%
\end{frame}%
}

\newdimen \boxbreite %wird dann gesetzt
\newcommand{\calcBoxBreite}[1]{% 1=wunschbreite
\boxbreite=#1%
\advance\boxbreite by -8pt % leer hier wichtig! rahmen bei rounded=true= 2x 4pt
%\the \boxbreite%
}

%calculate box width from given imagewidth
%\newdimen\boxbreite%
%\boxbreite=#1
%\advance\boxbreite by \boxrahmen % leer hier wichtig!
%\advance\boxbreite by \boxrahmen % leer hier wichtig!
%Ergebnis: \the\boxbreite

% tableofcontents-Options
% currentsection
% currentsubsection
% firstsection=⟨section number⟩ specifies which section should be numbered as section "1"
% hideallsubsections
% hideothersubsections
% part=⟨part number⟩ causes the table of contents of (part number) to be shown, instead of the
% table of contents of the current part
% pausesections pause before each secton
% pausesubsections pause before each subsecton
% sectionstyle=show/shaded  -> specifies how sections should be displayed. show,shaded,hide
% subsectionstyle=show/shaded/hide
% subsubsectionstyle

% DIES IST DIE URSACHE FÜR Acrobat Absturz!!!
% \AtBeginSection[] % Do nothing for \section*
% {%
% \einzeiler{\textcolor{white}{\bfseries \Large \thesection{}. \secname}}%kein plan warum textcolor entscheident ist...
% %\thesection{}.
% }
% \AtBeginSubsection[] % Do nothing for \subsection*
% {%
% \einzeiler{\textcolor{white}{\bfseries \large \thesection{}.\thesubsection{} \subsecname}}%kein plan warum textcolor entscheident ist...
% %\thesection{}.\thesubsection{}
% }

% Show the Outline at the beginning of every section
% \AtBeginSection[] % Do nothing for \section*
% {
% \begin{frame}[c]%
% \bfseries\Large%
% \begin{beamercolorbox}[wd=1.0\textwidth,ht=2.25ex,dp=1.0ex,center]{framehead}% ht=2.25ex,dp=1.0ex -> always same size
% %\thesection{}.
% \secname\phant
% \end{beamercolorbox}\par
% % \frametitle{Outline}
% %\tableofcontents[currentsection] %,hideallsubsections
% \end{frame}
% }

% \AtBeginSubsection[] % Do nothing for \subsection*
% {
% \begin{frame}[c]%
% \bfseries\large%
% \begin{beamercolorbox}[wd=1.0\textwidth,ht=2.25ex,dp=1.0ex,center]{framehead}%
% %\thesection{}.\thesubsection{}
% \subsecname\phant
% \end{beamercolorbox}\par
% \end{frame}
% }

% \einzeiler{\textcolor{red}{\bfseries \Large TODO: #1}}