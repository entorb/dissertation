\begin{frame}{Zusammenfassung}{}%leere Klammer wichtig, weil sonst nächste Zeile gefressen wird
{\bfseries Leitfähigkeit}\par
\begin{itemize}
\item \CrPd u. \WPd 
\\hohe \c bei kl. \C
\\\c$\downarrow$ bei gr. \C
%
\item \dmbi \\superlinear $\c(\C)$ \\ hohe \c bei gr. \C
\item \aob \\ starker Einfluss des Ausheizens
\end{itemize}
%
\vspace*{1ex}
{\bfseries Seebeck}\par
\begin{itemize}
\item n-Leitung auch bei gr. \C
\item monotone Abnahme \Pfeil \n $\uparrow$
\end{itemize}
%
\begin{textblock}{40}[1.0,0.5](118,30)% Bild ist 30 mm breit
\bildbox[scaled]{plot/MR-Cond-n-AS25+Pd30.pdf}{40mm}
\end{textblock}
%
\begin{textblock}{40}[1.0,0.5](118,70)% Bild ist 30 mm breit
\bildbox[scaled]{plot/MR-See+Es-n1234.pdf}{40mm}
\end{textblock}
\ton{\aob: Umformung nicht abgeschlossen}
\end{frame}

\begin{frame}{Zusammenfassung}{}
% \vspace*{6ex}
{\bfseries Beweglichkeit}\par
\begin{itemize}
\item \CrPd : $\mob>\cmVs{1}$
\item \CrPd u. \WPd: Abnahme bei gr. \C 
% \item thermisch aktiviert für
%  \\- \aob
%  \\- gr. \C von \CrPd u. \WPd
% \begin{itemize}
% \item \aob
% \item gr. \C von \CrPd u. \WPd
% \end{itemize}
\end{itemize}

\vspace*{1ex}
{\bfseries Dotiereffizienz}\par
\begin{itemize}
\item \CrPd : $\DopEff>20\percent$
\end{itemize}

\vspace*{1ex}
{\bfseries Modell konstanten Transportniveaus}\par
\begin{itemize}
\item realistische Werte und Trends
\end{itemize}

\vspace*{1ex}
% {\bfseries Fazit}\par
\begin{block}
{\bfseries Fazit}
\CrPd dotiert \CS am besten
\\\dmbi ist interessante Alternative da luftstabil
\end{block}
\end{frame}
% 

