%äöüß

% compress PDF, already default, so not needed any more
\pdfminorversion=5
\pdfobjcompresslevel=3 
\pdfcompresslevel=9

% Deutsche Sprache:
%\usepackage{ngerman} % ngerman, german
% Englische Sprache mit Deutschen Abschnitten: %
\usepackage[ngerman,american]{babel} % american , british
% Englische Sprache: %
% \usepackage[american]{babel} % american , british

% nun via \usepackage{tud-font-tm}
% Schriften von Hand wählen: Bsp= Palatino
% \renewcommand{\rmdefault}{ppl}
% \renewcommand{\sfdefault}{phv}
% \renewcommand{\ttdefault}{pcr}
% Sans Serif statt roman als default:
% \renewcommand*\familydefault{\sfdefault}

% Definitions
% =================
% Seperate paragraphs by an empty line instead of intending the line
% old:
%\setlength{\parskip}{2ex plus0.5ex minus0.3ex}
%\setlength{\parindent}{0pt}
% nun via KoMa Option parskip in DocClass

% put subsubsections in toc
\setcounter{secnumdepth}{3}
\setcounter{tocdepth}{3}

% layout of Captions in Tables and Figures (Bildunterschriften)
% \setkomafont{caption}{\small\sffamily}    % Font of the Caption \itshape
\setkomafont{caption}{\small\itshape}    % Font of the Caption \itshape
\setkomafont{captionlabel}{\bfseries}     % Font of the Label 'Figure' etc
\setcapindent{0em}                        % decrease intention of following lines
% \setcapwidth[c]{0.9\textwidth}            % NOTE: makes problems with package sidecap
% hence done manually in each figure using the following width:

\renewcommand{\topfraction}{.80} % up to 80% of page (from top) are allowed to be covered by a float until a float[t] page [p] is generated.

\renewcommand{\bottomfraction}{.80}% up to 40% of page (from bottom) are allowed to be covered by a float[b] until a float page [p] is generated.

% \renewcommand{\textfraction}{.2} %-> soviel muss text sein

\usepackage{amsmath}     % reqno: equation numbers right hand side of the eq
%\usepackage{amssymb} % \mathfrak does not word here :-( (font eufm10 missing)
%\usepackage{mathrsfs} % for \mathscr{ASDF}

\usepackage{geometry}           % Pagesettings
% Print: left=4cm
% online: left=3cm
% TODO: Print: Verlag Dr. Hut: left=2.7cm
% \geometry{left=2.7cm,textwidth=15cm,top=3cm,textheight=23.5cm}
\geometry{left=3.0cm,textwidth=15cm,top=3cm,textheight=23.5cm}

\newdimen\tmCapWidth
\advance\tmCapWidth by 1.0\textwidth

\usepackage[automark]{scrpage2}
%Clear
\ofoot[]{}
\cfoot[]{}
\ifoot[]{}
\ohead[]{}
\chead[]{}
\ihead[]{}

\setkomafont{pageheadfoot}{\small\sffamily}%\itshape

% \setkomafont{pageheadfoot}{\itshape} % klappt nicht, daher siehe unten
\setkomafont{pagenumber}{\small\sffamily}% aktivieren, falls die Seitenzahl kursiv soll

% \pagemark \headmark
\ihead[]{\headmark} % \itshape
\ohead[\pagemark]{\pagemark}

% \chead%
% [\footnotesize Stand: \the\day.\the\month.\the\year]
% {\footnotesize Stand: \the\day.\the\month.\the\year}
% \today -> EN format :-(

\pagestyle{scrheadings}

% Set Footnote symbols to symbols instead of numbers
\usepackage[perpage,symbol*]{footmisc} % perpage = reset counter on each page
% Latex Default = \DefineFNsymbols*{lamport}{*\dagger\ddagger\S\P|{**}{\dagger\dagger}{\ddagger\ddagger}}
\DefineFNsymbols*{TorbensFN}[math]{\ddagger\Sorg\P\dagger\ast{\dagger\dagger}{\ddagger\ddagger}{\ast\ast}{\Sorg\Sorg}{\P\P}}
% \|
% * wird nicht angezeigt und \S habe ich überschrieben, daher \Sorg
\setfnsymbol{TorbensFN}
% \setfnsymbol{wiley}

% http://web.slzm.de/blog/latex/schone-kapiteltitelseiten-in-latex/
\usepackage[table,dvipsnames,svgnames]{xcolor}
% \usepackage{blindtext}
% \usepackage{fix-cm}
\usepackage{titlesec}
% Style 1
%\renewcommand{\thechapter}{\Roman{chapter}}
\titleformat{\chapter}[display]
{\sffamily\bfseries}%\Large
{ %\Huge\textsc{\chaptertitlename}
\selectfont\color{gray}\hfill %lightgray
\begin{rotate}{90}
%\hspace{0.4cm}
\Large\selectfont \chaptertitlename
\end{rotate}
% \chaptertitlename \hspace*{0.0ex}
\fontsize{90}{70}\sffamily\textbf{\thechapter}}
{-2ex}%{-2ex}
{
\filleft\fontsize{30}{30}\selectfont\sffamily%\scshape
% \phantom{l}\phantom{g}invisible chars for fixing jumping of chapter name
}
[\vspace{0ex}\phantom{l}]

\usepackage[section]{placeins}
% defines the command \FloatBarrier: the name is program
% [section] : floats are not allowed to travel into another section

\usepackage[pdftex]{graphicx}
\usepackage{wrapfig}
% \begin{wrapfigure}{r}{21mm}%
% {\includegraphics{}\caption{}\label{}}%
% \end{wrapfigure}%
\usepackage[outercaption] {sidecap} % Caption besides the figure

\usepackage{pdfpages}
% \includepdf[pages=1-4,pagecommand={\thispagestyle{headings}}]{Meindoku.pdf}

% ===FONTS=== (vor siunitx laden!)

% %=== Variante 1: TUD CD ====
% % nicht VERGESSEN: set \sffamily after \begin{Document}
% \pdfmapfile{=univers.map} % von Hand laden, wenn map nicht registriert, muss in diesem Ordner liegen
% \usepackage[
% tudcd % TUD CD Universe
% %cm % Computer Modern
% %cmbright % CM-Bright fonts
% %helvet % Helvetiva
% %tx % Postscript txfonts
% %px % Postscript pxfonts
% %%lm % Latin Modern DOES NOT WORK WITHOUT FURTHER PACKAGES
% %,slantedGreek % -> slanted \ohm :-(
% %,T1experimental % = ???
% ]{tud-font-tm}
% % 2. Font DinBold manuell für Sections etc via \dinBold im Titel
% \pdfmapfile{=dinbold.map}
% \newcommand*{\dinBold}{\fontencoding{T1}\fontfamily{din}\fontseries{b}\upshape\selectfont}
% !!!TODO!!! for TU Font Universe
% \sisetup{obeyfamily=false,mathrm=mathsf,textrm=sffamily}
%\sisetup{detect-family = true} % for new versions of siunitx

%==== Variante 2: Charter + BeraSans====
% \usepackage{berasans} % = \renewcommand{\sfdefault}{fvs}
\usepackage{charter} % Bitstream Charter, mdbch
\usepackage[charter]{mathdesign} % replaces mathrsfs
\usepackage{beramono}
% Classico a.k.a. Optima as non-serif font
\pdfmapfile{=uop.map}
\renewcommand{\sfdefault}{uop} % Classico a.k.a. Optima

%==== Variante 3: Garamond & Classico Optima====
% \renewcommand{\rmdefault}{ugm} % Garamond
% \pdfmapfile{=uop.map}
% \renewcommand{\sfdefault}{uop} % Classico a.k.a. Optima
% \usepackage[garamond]{mathdesign}
% installed via
% http://my.opera.com/freedo/blog/2008/09/04/wie-kommt-die-garamond-ins-latex
% 1. getnonfreefonts -l
% 2. getnonfreefonts -d classico garamond
% 3. updmap

% === select font for headings
% \addtokomafont{title}        {\fontfamily{uop}}
% \addtokomafont{subtitle}     {\fontfamily{uop}}
% \addtokomafont{chapter}      {\fontfamily{uop}}
% \addtokomafont{section}      {\fontfamily{uop}}
% \addtokomafont{subsection}   {\fontfamily{uop}}
% \addtokomafont{subsubsection}{\fontfamily{uop}}
% könnte Probleme machen, weil umdefinierungen...
% besser nur sowas wie \normalfont \sffamily \mdseries

% Fonts VOR siunitx laden!
%\usepackage{textcomp} % Symbols like (C) etc, allows wring of °C and µ in text

% % http://www.mrunix.de/forums/archive/index.php/t-47484.html
% % mag diesen Satz + luximono statt courier
% \usepackage{berasans} % = \renewcommand{\sfdefault}{fvs}
% \usepackage{charter} % Bitstream Charter, mdbch
% \usepackage[charter]{mathdesign} % replaces mathrsfs
% \usepackage[scaled=1.1]{beramono}

% \usepackage{courier} % typewriter font
% % \usepackage{beraserif}
% % \usepackage{luximono} % typewriter font

%\usepackage{eulervm}

% % Palatino
% \usepackage{mathpazo}
% \usepackage[scaled=.95]{helvet}
% \usepackage{courier}

% % Times
% \usepackage{mathptmx}
% \usepackage[scaled=.92]{helvet}
% \usepackage{courier}

% \usepackage{bookman}

% \usepackage{newcent}

% http://my.opera.com/freedo/blog/2008/09/04/wie-kommt-die-garamond-ins-latex
% 1. getnonfreefonts -l
% 2. getnonfreefonts -d classico garamond
% 3. updmap

% \renewcommand{\rmdefault}{ugm} % Garamond
% \pdfmapfile{=uop.map}
% \renewcommand{\sfdefault}{uop} % Classico a.k.a. Optima
% \usepackage[garamond]{mathdesign}
% %\renewcommand{\ttdefault}{pcr}

%\usepackage{garamond}

% %\usepackage{dinbold}
% % \dinbold
% %\renewcommand*{\seriesdefault}{b}
% \pdfmapfile{=dinbold.map}
% \renewcommand{\sfdefault}{din}
% % \renewcommand{\bfdefault}{b}
% %\renewcommand{\rmdefault}{din}
% % \renewcommand{\ttdefault}{pcr}
% %\renewcommand{\mddefault}{l}

\usepackage{xfrac} % load before siunitx to allow for sfrac

\usepackage[
abbreviations = false
,noload={abbr} % \mm
,load={prefix,named,physical,accepted,addn} % no linebreaks or spacing here!
%prefix   \kilometer \milli
%named    \celsius \ohm \siemens
%physical \electronvolt
%accepted \degree \liter \percent
%addn     \BAR, \millibar, \angstrom
%,tabautofit
,tabnumalign=centredecimal % tabs: use S as collumn format -> center on '.'
,expproduct=tightcdot % cdot tightcdot times tighttimes
,per-mode=fraction
,fraction=sfrac
%,per=frac,fraction=nice % \per now produces nice fractions
%,fraction-function = \sfrac
,seperr=true
%,obeyfamily=true
% ,obeyall=true
% ,obeyitalic=true % for units in captions
%,obeybold=false
%,mathsrm=mathsf
%,mathssf=mathsf
%,mathstt=mathsf
% ,textrm=sffamily
% ,textsf=sffamily
%,texttt=sffamily
%,inlinebold=text
%,obeyfamily=true
,mode=text
% ,unitmode=text
% ,detect-display-math = true
% ,detect-family = true
% ,detect-mode = true
,detect-shape = true
% ,detect-weight = true 
% ,detect-all
]{siunitx} % Ubuntu 11.10 comes with version 1.3a (21.09.2009)

% \usepackage[obeyall]{siunitx}
% 
% \sisetup{
% %   detect-all
%   detect-family = true,
%   detect-inline-family = math
% % mode = math,
% % math-rm = \ensuremath
% }

% $ x = \SI{1.2e3}{\kg} $
% \SI{2.6(1)}{\meter} % error via ()

% \usepackage{nicefrac} % \nicefrac[]{1}{2} -> 1/2 in one symbol
% \nicefrac[]{1}{2} ---->  \sfrac{1}{2}

% \usepackage{stringstrings} \uppercase conversion etc

% \usepackage{oldgerm} % \textgoth, \textswab, \textfrak ; \gothfamily, \swabfamily, \frakfamily
\usepackage{yfonts} % \yinipar{L}etter
\usepackage{color}              % Support for colors :\textcolor{red}{Text}
% red, blue, white, green, cyan, magenta, yellow
\usepackage{xspace}             % \xspace can be used in newcommands without parameters to ensure whitespace after it

% \usepackage{setspace} % -> \singlespacing , \onehalfspacing , \doublespacing

% BibTeX
% =================
\usepackage[square,super,numbers,sort&compress]{natbib}     % needed by plainnat + dinat Bib Styls, square -> [] instead of () , numbers -> [1] etc

% Tables
% ===================
%\usepackage{float} % defines placement option H = exactly here
% generates scrbook Warning: \float@addtolists

\usepackage{booktabs} % -> \toprule \midrule \bottomrule
%\usepackage{array}              % nicer \hlines
\usepackage{longtable}

\usepackage{rotating} % for sidewaystable

% \usepackage{verbatim} %only for \verbatiminput ;-)

% load hyperref last solves the warning about Hfootnote:
% pdfTeX warning (dest): name{Hfootnote.1} has been referenced but does not exist , replaced by a fixed one
\usepackage[
pdftex % hypertex (default), pdftex
,unicode % Unicode encoded pdf strings
,final, % final, draft, debug -> more text in log file
% ,pdfauthor={\Verfasser}
% ,pdftitle={\Titel}
% ,pdfsubject={Dissertation, TU Dresden}
%,pdfcreator={}
%,pdfproducer={}
% ,pdfkeywords={Molecular Doping, Organic Semiconductors, Conductivity, Seebeck, Vacuum Deposition}
,pdfdisplaydoctitle={true} % title instead of filename
% ,pdfpagemode={UseOutlines} % UseOutlines, UseThumbs, UseNone, FullScreen
,pdfstartview={Fit}
,pdfpagelayout={SinglePage} % SinglePage, OneColumn, TwoPageLeft, TwoPageRight
%,pdfpagetransition={Blinds /Dm /V} % does not work in Linux
% ,pdftoolbar={false}
% ,pdfwindowui={false}
,bookmarks={true} % generate bookmarks
,bookmarksopen={true} % open bookmarks tree with depth=<bookmarksopenlevel>
,bookmarksopenlevel=1
,bookmarksnumbered={true}
,plainpages=false % behebt den Fehler "destination with the same identifier"
,citebordercolor={0 1 0}
,filebordercolor={0 0.5 0.5}
,linkbordercolor={1 1 0} % 1 1 0 = yellow
,urlbordercolor ={0 0 1}
,pdfborder={0 0 0}, % 0 0 0: no colorful frames around links, comment out if wanted
,colorlinks={false} % color for text of links (replaces link border frames)
,linkcolor={blue}
,anchorcolor={black}
,citecolor={green}
%,filecolor={black}
,urlcolor={blue}
]{hyperref}

% Symbols instead of numbers for footnotes
%\renewcommand{\thefootnote}{\fnsymbol{footnote}}

% this is taken from l2tabu.pdf:
\tolerance 1414
\hbadness 1414
\emergencystretch 1.5em
\hfuzz 0.3pt
\widowpenalty=10000
\vfuzz \hfuzz
\raggedbottom

%\hyphenpenalty=500               % penalty for words split on two lines

\usepackage{ifthen}
%\ifthenelse{\equal{\tmDraft}{True}}%
%{}% True
%{}% False