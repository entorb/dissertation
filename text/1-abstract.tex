\addcontentsline{toc}{section}{Abstract / Kurzfassung}
\vspace*{2cm}
\section*{Abstract}
This work aims at improving the understanding of the fundamental physics behind molecular doping of \OSCs, being a requirement for efficient devices like organic light-emitting diodes (OLED) and organic photovoltaic cells (OPV). The underlying physics is studied by electrical conductivity and thermoelectrical Seebeck measurements and the influences of doping concentration and temperature are investigated. Thin doped layers are prepared in vacuum by thermal co-evaporation of host and dopant molecules and measured \insitu.

The fullerene \CS, known for its high electron mobility, is chosen as host for five different n-dopants. Two strongly ionizing air-sensitive molecules (\CrPd and \WPd) and three air-stable precursor compounds (\aob, \dmbi and \meodmbiI) which form the active dopants upon deposition are studied to compare their doping mechanism.
High conductivities are achieved, with a maximum of \Scm{10.9}.
Investigating the sample degradation by air-exposure, a method for regeneration is proposed, which allows for device processing steps under ambient conditions, greatly enhancing device fabrication possibilities.

Various material combinations for p-doping are compared to study the influence of the molecular energy levels of host (\meo and \lili) and dopant (\FS and \CSF). Corrections for the only estimated literature values for the dopant levels are proposed.
Furthermore, the model system of similar-sized host \pen and dopant \FV is studied and compared to theoretical predictions.

Finally, a model is developed that allows for estimating charge carrier mobility, \nLong, \DopEffLong, as well as the \EtLong position from combining conductivity and Seebeck data.
\newpage
\vspace*{2cm}
\section*{Kurzfassung}
\begin{otherlanguage}{ngerman}
Diese Arbeit untersucht organische Halbleiter und den Einfluss von molekularer Dotierung auf deren elektrische Eigenschaften, mit dem Ziel effizientere Bauelemente wie organische Leuchtdioden oder Solarzellen zu ermöglichen.
Mittels Leitfähigkeitsuntersuchungen sowie thermoelektrischen Seebeck-Messungen werden die Einflüsse der Dotierkonzentration sowie der Temperatur auf die elektrischen Eigenschaften dünner dotierter Schichten analysiert.
Das Abscheiden der Schichten durch Koverdampfen im Vakuum ermöglicht eine \insitu Analyse.

Das Fulleren \CS, bekannt für besonders hohe Elektronenbeweglichkeit, wird als Wirt für fünf verschieden n-Dotanden, zwei extrem stark ionisierende luftreaktive (\CrPd und \WPd) sowie drei luftstabile (\aob, \dmbi und \meodmbiI), verwendet. Dies ermöglicht Schlüsse auf die unterschiedlichen zugrunde liegenden Dotiermechanismen und das Erreichen von Leitfähigkeiten von bis zu \Scm{10.9}.
Für einen der luftreaktiven Dotanden wird die Probendegradation an Luft untersucht und eine Regenerationsmethode aufgezeigt, die Prozessierungsschritte in Luft erlaubt und somit entscheidend für zukünftige Bauelementfertigung sein könnte.

Verschiedene p-dotierte Materialkombinationen werden untersucht, um den Einfluss der molekularen Energieniveaus von Wirt (\meo und \lili) und Dotand (\FS und \CSF) auf die Dotierung zu studieren. Dies ermöglicht Schlussfolgerungen auf die in der Literatur bisher nur abgeschätzten Energieniveaus dieser Dotanden.
Ferner werden die Eigenschaften des bereits theoretisch modellierten Paares Pentacen und \FV mit den Vorhersagen verglichen und die Abweichungen diskutiert.

Abschießend wird ein Modell entwickelt, das die Abschätzung von Dotiereffizienz, Ladungsträgerkonzentration, Ladungsträgerbeweglichkeit sowie der Position des Transportniveaus aus Leitfähigkeits- und Seebeck-Messungen erlaubt.
\end{otherlanguage}