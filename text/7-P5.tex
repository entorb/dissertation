%äöüß
\chapter{Pentacene p-Doped by \FV}\label{chap:P5}
\addcontentsline{lof}{chapter}{\thechapter\hspace*{1ex} Pentacene p-Doped by \FV}

%
\intro{After studying two amorphous organic hole transporters in the previous chapter, the high mobility prototypical polycrystalline organic material \pen is investigated in the following. \FV is chosen as p-dopant, since it has almost the same size and weight as \pen and is expected to neatly integrate into \pen layers at low \CLongs.
Theoretical studies predict an increasing doping efficiency with \CLong for this material combination, which is experimentally tested in this chapter.
}

\newpage
%
The combination of \pen doped by \FV has been studied earlier by several groups and techniques. AFM surface scans have shown that layers of \pen can be grown highly crystalline and that at very low \CLong, \FV does not disturb the molecular order of \pen\cite{Ha2009}. Increasing the \CLong, the crystallite size has been found to decrease and rough surface structures have been reported\cite{Kleemann2012a}.
Details about the materials are summarized in the materials \secref{Mat}.

Theoretical studies on this model system of \pen doped by \FV, with similar-sized dopant and host, suggest\cite{Mityashin2012a} that for \OSCs, a certain threshold \CLong exists, below which doping does not increase the conductivity. This threshold is attributed to electron--hole attraction hindering the charge pair dissociation. Increasing the \CLong \C, the potential landscape of the \IE IE is expected to be altered such that percolation pathways for dissociation are generated. Hence, the doping efficiency is expected to increase with \C.
%
To test this model, conductivity and Seebeck investigations are performed.

\section{Conductivity Changes after Preparation}%
\label{sec:ResP5-LongTimeCond}
In an analogous way to the previously investigated materials, the change of conductivity of freshly produced \nm{30} thick layers over time is investigated before reporting on the observed conductivities.
%
\cBild[b]
{MR-LongTimeCond-P5}
{Conductivity change during first hour after preparation, fitting parameters}
{Conductivity change during the first hour after sample preparation for \pen doped by \FV. Fitting parameters, according to \eqnrefPage{LongTimeCond-exp}.
}
%
As for most sets of materials, the conductivity of each sample is continuously probed \insitu for 1~hour at a fixed temperature of \T[25], as discussed in detail in \secref{ResPdLongTimeCond}. A strong reduction of conductivity \c over time is detected for most samples of \pen doped by \FV. To quantify the change in conductivity, the data are fitted according to \eqnrefPage{LongTimeCond-exp} and a good agreement with the fit function is found for most samples. The resulting fitting parameters are presented in \figref{MR-LongTimeCond-P5}.

The strongest reduction is determined for the lowest doped sample, resulting in a fitted maximal relative change $\chi$ of $-50\,\%$ of the initial value of the conductivity. Samples of higher \CLongs show smaller decreases and two samples (\C[0.043] and \mr{0.066}) even a slightly increasing \cLong over time. The highest doped sample of \C[0.193] shows again a strong reduction.

The time constant $\tau$ scatters around \SI{30}{\minute} for all samples, being comparable to the results for \CS n-doped by \dmbi (compare \figrefPage{MR-LongTimeCond-n-AS-fitparameter}).

\section{Relation of Conductivity to Doping Concentration}%
\label{sec:ResP5-CondMR}
%
\cBild
{MR-Cond-P5}
{As-prepared conductivity}
{As-prepared conductivity \vs \CLong of layers of \pen doped by \FV, probed at \T[25], directly after sample preparation (full circles) and after thermal annealing (open circles). Literature values taken from Harada\etal\cite{Harada2010} and Kleemann\etal\cite{Kleemann2012a} are added for comparison.
Dashed lines are linear fits with slopes 1.0 and 2.5.
}

Measured directly after sample fabrication at \T[25], the detected conductivities are in the range of \c[5.8E-3] to \Scm{3.7E-2}, and a sublinear (slope $< 1.0$) increasing with \CLong up to \mr{0.080}, followed by a decrease at higher \C is found, as shown by the full circles in \figref{MR-Cond-P5}.
The sublinear rise of $\c(\C)$ suggests a reducing charge carrier mobility with increasing \CLong and hence a compensation to the gain of \nhLong induced by the dopants. This decrease of the mobility is attributed to a disturbance of the polycrystalline morphology of \pen by the rising number of dopants, as reported in literature\cite{Kleemann2012a}.

As described in the previous section, the \insitu conductivity strongly changed after sample fabrication. To reach stable measurement conditions, the samples were thermally annealed for 1~hour at \T[70] prior to further investigations. Afterwards, the conductivities are measured again and found to be constant over time, thus the heating seems to accelerate and saturate the effect responsible for the change of the \cLong over time. The values are displayed by the open circles in \figref{MR-Cond-P5}. It can be seen that the heat treatment changes the room temperature conductivity in the same direction as the trend of the continuous investigations after sample preparation indicated.
The conductivity of the lowest doped sample drops by two orders of magnitude, whereas samples of higher \CLongs show less reduction. Only the \cLong of the sample of \C[0.043] increased slightly after heating. Two samples with higher \CLongs show almost constant \cLongs and the highest doped sample decreased by almost one order of magnitude, in agreement with the tendency of the continuous measurements directly after sample fabrication.

After thermal annealing, the \cLong has a strong \C-dependency in the low to medium doping regime. A strongly superlinear rise with a slope around 2.5 in this double-logarithmic plot is found for the four lowest doped samples, as indicated by the dashed line in \figref{MR-Cond-P5}. Similar slopes are determined for \meo and \lili p-doped by \CSF in \secref{ResP-CondMR}, but at large \CLongs.
%
The strong change of the \cLong directly after fabrication as well as after thermal annealing is either attributed to changes of the mobility or to changes of the \nhLong, for example by diffusion and agglomerating or re-evaporation of the light diffusive dopant \FV, having a sublimation temperature in the range of \Tdep[100]. As this effect has a \C-dependency, it might be responsible for the superlinear rise of $\c(\C)$ as well.
The changes of the \cLong are smallest for samples of \CLongs between \C[0.043] and \mr{0.066}, indicating that \FV is best integrated into the layer of \pen in this range of \CLong.

Interestingly, the conductivity values as well as their slope prior to annealing are in good agreement with investigations published by Harada\etal\cite{Harada2010}, whereas after heating they agree to experiments by Kleemann\etal\cite{Kleemann2012a}. Both sets of data are shown in \figref{MR-Cond-P5}, with the authors' kind permissions.
%
Harada\etal have reported\footnote{Personal correspondence with the author.}
that they observed a decreasing conductivity after sample processing as well. They decided to measure conductivity and Seebeck coefficient as quickly as possible, instead of performing a thermal annealing step as done here. This explains why their conductivity data are in agreement with the above discussed measurements prior to heat treatment. Kleemann\etal on the other hand used samples that where in contact with air for 30~minutes after fabrication. Thus, this suggests that air-exposure has a similar influence on the conductivity of these samples than thermal treatment. It is possible that the light and volatile \FV molecules desorb from the layer upon contact with air or during heating in vacuum.

Both studies have presented \C-dependent OFET-mobility measurements in addition to \cLong studies and reported a similar trend of a slowly decreasing OFET-mobility at low \C, followed by a significant decrease at \Cgr{0.050}. Hence, the drop of \c at high \C detected in all four experiments and shown in \figref{MR-Cond-P5}, is explained by a decreasing mobility.
Due to very narrow contact distances and thus high fields in an OFET geometry, it is expected that the \nhLong is strongly enhanced and hence compensates for dopant induced traps. Thus, in conductivity geometry a different tendency of the mobility, especially for low \CLongs, is expected.

\section{Comparison of Seebeck Energy and Activation Energy}
\label{sec:ResP5-EsEact}

\cBild[b]
{MR-Es+Eact-P5}
{Comparing \Eact and \Es}
{Comparing \EactLong \Eact and \EsLong \Es. \Eact is fitted from conductivity data of \T[25] to \grad{70}, \Es is measured at \Tm[40]. Literature values for \Es taken with kind permission from Harada\cite{Harada2010} are added for comparison.
}

Temperature-dependent conductivity investigations in the range of \T[25] to \grad{70} on this material system allow to derive an \EactLongL for each sample, using \eqnrefPage{CondActivation}. The derived \Eact is depicted in \figref {MR-Es+Eact-P5} and found to strongly vary with \CLong, with a maximum of \Eact[357] at \C[0.011]. The obtained trend of $\Eact(\C)$ is almost inverse to the tendency of the $\c(\C)$ after thermal annealing. Samples of high \c show low values of \Eact and vice versa, in agreement with the data presented in the previous chapters and only the lowest doped sample deviating.

Besides conductivity investigations, Seebeck measurements at \Tm[40] (after thermal annealing) are performed on this material system as well, allowing to derive the energetic difference \Es between \EfLongL and \EtLongL.
\Es is positive for all samples, as expected for p-doped layers and the values are presented in \figref{MR-Es+Eact-P5} where they are compared to \Eact.
The highly doped samples yield a slowly decreasing $\Es(\C)$, in the range of \Es[103] to \meV{73}, whereas for the lowest doped sample the Seebeck measurement was not successful.
This rather small \Es at low \C indicates a high doping efficiency of \FV in \pen. Another explanation for the low \Es is that intrinsic \pen has been reported to have an extremely narrow \dosLongL with a Gaussian width in the range of only \gausswidth[70]\cite{Yogev2011}. Such a narrow \dos requires a small \Es to generate free charges.
Upon doping, the width of the \dos and hence \Es are expected to rise due to molecular disorder. This effect is not visible in the data and might be compensated by the increasing \CLong, resulting in an almost constant \Es.

The differences between \Es and \Eact for most samples are attributed to a thermal activation of the mobility, as discussed in \secref{ResPd-EsEact}. This contribution is strongest for the sample of \C[0.011] and decreasing with \C. As in the same \C-regime, a superlinear rise of $\c(\C)$ is observed and \Es is almost constant, the decreasing thermal activation of the mobility seems to be correlated to a mobility rise in this regime of \C.

Harada\etal\cite{Harada2010} have performed Seebeck measurements on \pen doped by \FV as well, but at lower \Tm[24] and without thermal annealing the samples. Their data are included in \figref{MR-Es+Eact-P5}.
An almost identical value of \Es is measured at \C[0.020] in both setups: Harada reported \Es[81.7] (\S[275.0] at \Tm[24]), compared to \Es[82.2] (\S[262.5] at \Tm[40]) measured in our setup.
At higher \CLongs, the two experiments differ, with Harada's values being lower.
The deviation of the two experiments is attributed to the fact that Harada did not anneal the samples prior to Seebeck investigations, as discussed above. Thus, this deviation is an indication for heating or air-exposure of highly \FV-doped \pen samples reducing the \nhLong, most probably by the agglomeration or re-evaporation of the diffusive \FV molecules. At low \CLongs, where the two Seebeck measurements are in agreement, the \nhLong seems not to be affected by thermal annealing and therefore the different conductivities are attributed to changes in the mobility induced by morphological changes in the layer and accelerated by the heat treatment.
%
Re-evaporation of dopants is not likely to be present in the data at low \C, since this would correspond to a shift of \Es to smaller \CLongs, flattening the curve even more.

\section{Conclusion}
%
The strong reduction of the \insitu \cLong over time, which could be accelerated and saturated by heat treatment, is attributed to morphological changes in the layer, reducing the mobility. A decreasing \cLong at high \CLongs is in agreement with a strongly reduced OFET-mobility reported in literature\cite{Harada2010,Kleemann2012a}.
%
The rather small \Es at low \CLongs indicates a high doping efficiency of \FV in \pen and/or a narrow \dosLong. At higher \CLongs the \Es is found to be almost constant, but larger than obtained for highly n-doped \CS samples, which is attributed to a broadening of the \dosLong, partly compensated by the increasing \CLong.
A comparison with literature values of unannealed samples suggests that at \Cgr{0.020} during thermal treatment, re-evaporation of the light and diffusive dopant molecules occurs.

% Model of Mityashin2012a
Neither the presence of a threshold \CLong for the generation of free charge carriers nor indications for an increasing doping efficiency upon rising \CLong, as predicted by Mityashin\etal\cite{Mityashin2012a}, are observed by the measurements.
Rather high conductivities directly after sample fabrication, showing only a moderate increase with \C, even contradict this model. Seebeck measurements do not show indications for an increasing doping efficiency with \C, which would be correlated to a more rapid decrease of \Es.
It is possible that the predicted phenomena are concealed by morphological effects or that they occur at lower \CLongs.
