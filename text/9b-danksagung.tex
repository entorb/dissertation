% \@ifpackageloaded{lineno}{\nolinenumbers}{}
\chapter*{Acknowledgments / Danksagung}
\addcontentsline{toc}{chapter}{Acknowledgments / Danksagung}
%
\begin{otherlanguage}{ngerman}
% \swabfamily%
% \gothfamily%
\frakfamily%
\fraklines% optimized better linespacing
% ``
%
\yinipar{N}un möchte ich mich bei all den Leuten bedanken, die diese Arbeit auf die ein oder andere Art unterstützt haben.
%
Zuallererst danke ich meinem Doktorvater Professor Dr. Karl Leo für die Möglichkeit, diese Dissertation anzufertigen und dafür, daß er das: Institut für Angewandte Photophysik zu dem gemacht hat, was: es: heute ist.\\
%
Professorin Dr. Elizabeth von Hauff danke ich für die bereitwillige Begutachtung dieser Arbeit. 

\yinipar{B}esonderer Dank gebührt meinem Gruppenleiter Moritz Riede, der die durch Konversion von Frustration in Motivation diese Arbeit stets: unterstützte.
Hoch anzurechnen ist ihm auch, daß er das: Ringen um Projektmittel weitestgehend von den Schultern seiner osol-Gruppe fern hielt. Dies:, kombiniert mit seiner Affinität für technische Details:, bescherte ihm viele 36h Tage, was: ihn aber trotzdem nicht davon abhielt, Zeit für das: Korrekturlesen von Arbeiten und Veröffentlichungen zu finden.

\yinipar{I} thank Debu, Debdutta Ray, who claimed that my early state data were making sense after all and supported this: work by many fruitful discussions: and numerous: private lessons: in semiconductor physics:, as: well as: troubleshooting the setup and proofreading this: manuscript.\\
Mein zweiter wichtiger Diskussionspartner und Ideenfundus: Hans: Kleemann saß praktischerweise im selben Büro wie Debu, was: zu einem erheblichen Schokikonsum führte, der nur noch von seinem eigenen Kaffeedurchsatz getoppt wurde. Ihm danke ich auch für das: Schleifen meines: Schreibstils:.
Ohne Debu und Hans: hätte diese Arbeit sicher nicht diese Form erreicht.

\yinipar{I} enjoyed the fruitful cooperation with Professor Zhenan Bao's: group of Stanford University on the dmbi n-dopants: and was: glad for the chance to meet Zhenan Bao, Peng Wei and Benjamin Naab in person.\\
Ebenfalls: danke ich der Novaled AG und insbesondere Jan Blochwitz-Nimoth für die Bereitstellung der n-Dotanden.

\yinipar{D}as: Handwerk des: Dotierens: durch Koverdampfung lernte ich maßgeblich von Selina Olthof, die auch ups-Messungen beisteuerte und mit der ich den Interkontinentalvideoabend erfunden habe.
Ich bitte um Verzeihung dafür, daß ich dich in der Danksagung der ersten Veröffentlichung vergessen habe.
% \vspace*{1em}

\yinipar{M}einem alten Weggefährten Jan Meiß danke ich dafür, daß er mich für das: iapp rekrutierte, diverse Teilnahmen an Teamlaufveranstaltungen organisierte und eine Papervorlage mit Grundstruktur und Mietzekatzen entwarf.
\vspace*{1em}

\yinipar{V}iel Spaß hatte ich mit meinem langjährigen Bürokollegen Johannes: Widmer, dessen angenehme Art einen steten Ruhepol bildete. Ich genos:s: die unzähligen technischen Diskussionen und freute mich über kritische Korrekturen der Grundlagenkapitel. Seine vorbildliche Denkweise und sein Engagement beeindruckten mich immer wieder.

\yinipar{E}inige Leute haben freundlicherweise Messungen für diese Arbeit beigesteuert.
Da wären ups-Messungen von Selina Olthof und Max Tietze, ofet-Messungen von Moritz Hein und Jens: Jankowski, sowie Fabrikation und Vermessung von organischen Solarzellen (die es: leider nicht in diese Arbeit geschafft haben) durch Bernhard Siegmund, Jens: Jankowski, Caroline Walde und Ralph Dannroth. Für die Anleitung und Fehlersuche bei afm-Messungen danke ich Lars: Müller-Meskamp, Sarah Röttinger sowie Tobias: Mönch.
Hans: Kleemann und Kentaro Harada stellten freundlicherweise ihre Daten für das: Pentacen Kapitel zur Verfügung. Ken danke ich außerdem für die Übergabe des: Messaufbaus: und insbesondere der Vorführung der japanischen Lösung für die Stickstoffkühlung.
Axel Fischer, Max Tietze und Florian Wölzl danke ich für die gute Zusammenarbeit im Frankensteinprojekt.
Andreas: Wendel präparierte Grundkontakte auf alle in dieser Arbeit verwendeten Substrate in Massenproduktion und ersparte mir dadurch viel Zeit.

\yinipar{P}rofessor Dr. Horst Hartmann, Markus: Hummert, Roland Gresser und Alexander Hoyer danke ich für die Diskussionen über die chemischen Strukturen der Dotanden sowie Auffrischung meiner Chemiekenntnisse. Gerne diskutierte ich auch mit Max Tietze, Matthias: Schober, Janine Fischer, Johannes: Widmer und Paul Pahner über physikalische Modelle und freute mich über Anmerkungen zu meinen Veröffentlichungs:entwürfen von Björn Lüssem, Martin Pfeiffer und Jan Blochwitz-Nimoth.\\
Über Fragen der zukünftigen Energieversorgung habe ich mich gerne mit Carsten Knoll, Christoph Schünemann, Johannes: Widmer, Jan Meiß und Wolfgang Tres:s: ausgetauscht.

\yinipar{F}ür technische Hilfe bei der Erweiterung des: Messaufbaus: danke ich
Carsten Wolf (Rat und Tat sowie Ölpestbekämpfung), Sven Kunze , Daniel Dietrich und Daniel Kasemann, sowie Carsten Knoll und Philipp Latzel für Tipps: zur Pythonprogrammierung. Ferner den ufo-Piloten für das: Nachsehen
Christiane Falkenberg, Selina Olthof, Christian Körner, David Wynands:, Felix Holzmüller, Franz Selzer und Toni Müller.
Sowie Magdalene Menke für die Bereitstellung der Lampe für die Beleuchtung der aob-Proben während der Präparation.
Auch dem unbekannten Opensource Softwareentwickler sei an dieser Stelle gedankt, da ich seine Produkte gern und viel genutzt habe, z.B. Linux, Latex, Perl, Python, Gnuplot, LibreOffice, Inkscape und Gimp.

%Besonderer Dank gebührt auch den Freunden, die beim Ausmerzen der Tippfehler geholfen haben.
Beosnedrer Dnak gehübrt acuh den Ferudnen, die biem Aus:mrezen der Tpipfheler gehlofen haebn.

\vspace*{1em}

\yinipar{V}eelen Dank an all de bannig veelen Lüd, de mi in jede Lag wiegerholpen hebbt. Alleen har ick dat
nich so öllig tosamen kreegen. Ik denk girn trüch an:
\vspace*{2em}
%
\begin{itemize}
% die fröhliche Atmosphäre mit den Bürokollegen
\item de fründliche Ümgang mit de Lü in de Schrievstuv:
Johannes: Widmer, Melanie Lorenz-Rothe, Roman Forker, Björn Lüssem und Christian Wagner (mit seinem beeindruckenden Schreibtischchaos)
%
% die gute Zusammenarbeit im Labor
\item de gaude Tosamenholt in de Warkstäh:
Benjamin Röttig, Danny Jenner, Daniel Kasemann, Jan Förster, Sylvio Schubert und Tobias: Günther
%
% den vielen Spaß mit den Büronachbarn
\item dan veelen Spaß mit de Nahbers: vun uns: Schrievstuv:
Christoph Schünemann, Debdutta Ray, Hans: Kleemann, Jörg Alex, Lorenzo Burtone und Philipp Siebeneicher
%
% die Beherbergung während der Mittagspausen in meinem Zweitbüro durch
\item de Ünnerkunft in de Middagsstunn in mien tweite Schrievstuv:
Hannah Ziehlke, Jan Meiß, Steffen Pfützner, Maik Langner, Robert Brückner, Daniel Kasemann und Johannes: Haase.\\
Eten un drinken hölt Liev un Seel tosamen
%
% schöne Konferenzreisen mit
\item dat scheune ut feuern tau Konferenzen in de wiege Welt mit:
Caroline Murawski, Christian Körner, Christian Uhrich, Gregor Schwartz, Hans: Kleemann, Lars: Müller-Meskamp, Lorenzo Burtone, Marion Wrackmeyer, Merve Anderson, Michael Machala und Tobias: Schwab
%
% Sport
\item Sport mit de Kollegen, Hol di fuchtig un risch as: dull as: du kanns:{}:
% sportlichen und spaßlichen Höchstleistungen mit den Sportsfreunden des: iapp:
Jan Meiß, Steffen Pfützner, René Kullock, Nico Seidler, Robert Brückner, Tobias: Schwab, Phil Goldberg, Moritz Riede (Laufen, Triathlon), Christiane Falkenberg, Christoph Schünemann, Janine Fischer, Hans: Kleemann (Akrobatik), Hannah Ziehlke (Fit im Büro), Felix Holzmüller, Matthias: Schober sowie Hans-Georg von Ribbeck und Roland Gresser (sportliches: Radeln und so)
%
\item dat Rümvriemeln in veele scheune Bastelstund'n:
% viele Bastelstunden mit
Christiane Falkenberg, Daniel Kasemann, Hans: Kleemann, Johannes: Widmer und Selina Olthof
%
% die lieben, bisher noch nicht erwähnten Kollegen
\item un nich to vergeten, de leiben Lü, de ik reinweg noch nich upschreben hev:
Alexander Kittner, André Döring, André Merten (Projektberichte), Annette Petrich (Hilfe im Chemielabor), Chris: Elschner, Christoph Sachse, Ellen Siebert-Henze, Hannes: Klumbies:, Ines: Rabelo de Moraes:, Marek Rölke, Marieta Levichkova, Martin Hermenau, Rico Meerheim, Rico Schüppel, Ronny Timmreck, Ruben Seifert, Simone Hofmann, Steef Corvers:, Stefan Auschill, Till Hoheisel und Tina Träger sowie den leider hier vergessenen Mitgliedern der osol-Gruppe für die gute Zusammenarbeit.
%
\end{itemize}

%Neben den Kollegen danke ich meiner Familie und meinen Freunden für die Unterstützung und insbesondere für die Ablenkungen und Aufmunterungen, wenn es: im Labor mal wieder nicht so lief, wie geplant, oder wenn das: Schreiben besonders: zäh von Statten ging.
% \yinipar{N}
\yinipar{N}äm de Kollegen müch ick aber ok miene Familje, Franca un Frünn Dank seggen. Sei hebt mi mit veel Pläsier und Tauspruch über de ganze Tied ünnerstützt. Franca dank ick besünners för de Nahhülp in Saaken Work-Life-Balance.
% för de Ünnerstützung
% de Öllern
% för miene leiben Öllern
Ick dank miene Frünn för de Aflenkungen mid Tühnkrom un Spijöks:, wenn dat in de Warkstäh nich so an loopen weu as: dacht oder wenn bi dat Schrieven de Faden afreten weu.

Veelen herzlichen Dank dörfür, dat weu ne gaude Tied!

\end{otherlanguage}

