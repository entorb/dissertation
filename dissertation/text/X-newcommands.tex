%äöüß

% Shortcuts:
% STRG + Shift + V = Verbatim
% STRG + Shift + E = Equation

% === to do markup ===
\newcommand{\todocite}  [0] {%
\todof{Reference to be added}
}
\newcommand{\todo}[1]{%
\textcolor{red}{TODO: #1}\xspace%
}
\newcommand{\todof}[1]{%
\textcolor{red}{\footnote{\textcolor{red}{TODO: #1}}}%
}%
\newcommand{\note}[1]{%
\textcolor{red}{\footnote{\textcolor{green}{NOTE: #1}}}%
}
\newcommand{\wort}[1]{%
\textcolor{violet}{#1}%
}
\newcommand{\klau}[1]{\textcolor{blue}{#1}}
\newcommand{\klauPd}[1]{\textcolor{magenta}{#1}}

% param1 = label
% param2 = caption
\newcommand{\todoplot}[2]{%
\begin{figure}%
\centering%
\includegraphics{fig/dummy.pdf}%
\caption{\todo{#2}}%
\label{fig:#1}%
\end{figure}%
}
\renewcommand{\todoplot}[2]{\todo{Plot #2}}

\newcommand{\was}[1]{%
\textbf{#1}\\%
}

% === TEXT ===

\newcommand{\intro}[1]{%
%\emph{#1}%
\textit{#1}%
}

\newcommand{\entdecker}[4][\empty]%
{\footnote{%
after #3 scientist \name{#2}%
\ifthenelse{\equal{#1}{\empty}}%
{}%
{\cite{#1}}%
\,(#4)%(1900-1950)
}}

\newcommand{\name}[1] {#1}% \textsc

% bei Verwendung von TU Schriften wird im mathemodos ein falsches % zeichen verwendet
% \renewcommand{\percent}{
% \ifmmode%
% \text{{\%}}%\textsl\textit
% \else%
% \%%
% \fi%
% }

% von http://www.mrunix.de/forums/showthread.php?t=45035
\let\exporg\exp % damit keine endlosschleife erzeugt wird
\renewcommand{\exp}[1]{\exporg{\left(#1\right)}}

% Torbens "draft" mode: newpage at each (sub)section
% \let\sectionorg\section % damit keine endlosschleife erzeugt wird
% \renewcommand{\section}[1]{\newpage\sectionorg{#1}}
% \let\subsectionorg\subsection
% \renewcommand{\subsection}[1]{\newpage\subsectionorg{#1}}

% === UNITS ===

% [per=frac,fraction=nice,valuesep=thick]
\newcommand{\K}[1] {\SI{#1}{\kelvin}}
\newcommand{\grad}[1] {\SI{#1}{\celsius}}
\newcommand{\nm}  [1] {\SI{#1}{\nano\meter}}
\newcommand{\cmVs}  [1] {\SI{#1}{\centi\meter\squared\per\volt\per\second}}
\newcommand{\um}  [1] {\SI{#1}{\micro\meter}}
\newcommand{\mm}  [1] {\SI{#1}{\milli\meter}}
\newcommand{\meV} [1] {\SI{#1}{\milli\electronvolt}}
\newcommand{\eV}  [1] {\SI{#1}{\electronvolt}}
\newcommand{\mV}  [1] {\SI{#1}{\milli\volt}}
%\newcommand{\mbar}[1] {\SI{#1}{\milli\bbar}}
\newcommand{\pa}[1] {\SI{#1}{\pascal}} % E-7 mbar = E-5 Pa
\newcommand{\uVK} [1] {\SI{#1}{\micro\volt\per\kelvin}} % \micro -> Font eurm10 at 10pt not found
\newcommand{\Scm} [1] {\SI{#1}{\siemens\per\centi\meter}}
\newcommand{\mr}[1] {\SI{#1}{MR}}
\newcommand{\wt}[1] {\SI{#1}{wt\percent}}
\newcommand{\sek}[1] {\SI{#1}{\second}}
%\newcommand{\minute}[1] {\SI{#1}{\minute}}
\newcommand{\gMole}[1] {\SI{#1}{\gram\per\mole}}
% \newcommand{\As}  [1] {\SI{#1}{\angstrom\per\second}}

\let\uorg\u
\renewcommand{\u}[1] {\SI{#1}{\atomicmass}}

% === VARIABLES+UNITS ===

% T or T=50 C
\newcommand{\T} [1][\empty]%
{%
\ifthenelse{\equal{#1}{\empty}}%
{\ensuremath{T}\xspace}%
{\mbox{\ensuremath{T=\grad{#1}}}}%
}

\newcommand{\Tm} [1][\empty]%
{%
\ifthenelse{\equal{#1}{\empty}}%
{\ensuremath{T_\text{m}}\xspace}%
{\mbox{\ensuremath{T_\text{m}=\grad{#1}}}}%
}

% glass transition Temp
\newcommand{\Tg} [1][\empty]%
{%
\ifthenelse{\equal{#1}{\empty}}%
{\ensuremath{T_\text{g}}\xspace}%
{\mbox{\ensuremath{T_\text{g}=\grad{#1}}}}%
}
% deposition Temp
\newcommand{\Tdep} [1][\empty]%
{%
\ifthenelse{\equal{#1}{\empty}}%
{\ensuremath{T_\text{dep}}\xspace}%
{\mbox{\ensuremath{T_\text{dep}=\grad{#1}}}}%
}
% substrate Temp
\newcommand{\Tsub} [1][\empty]%
{%
\ifthenelse{\equal{#1}{\empty}}%
{\ensuremath{T_\text{sub}}\xspace}%
{\mbox{\ensuremath{T_\text{sub}=\grad{#1}}}}%
}
\newcommand{\V} [1][\empty]%
{%
\ifthenelse{\equal{#1}{\empty}}%
{\ensuremath{V}\xspace}%
{\mbox{\ensuremath{V=\SI{#1}{\volt}}}}%
}

\newcommand{\cLong} [0]{conductivity\xspace}
\newcommand{\cLongL} [0]{conductivity \ensuremath{\sigma}\xspace}
\newcommand{\cLongs} [0]{conductivities\xspace}
\newcommand{\cLongsL} [0]{conductivities \ensuremath{\sigma}\xspace}
\let\corg\c % Backup
\renewcommand{\c} [1][\empty]%
{%
\ifthenelse{\equal{#1}{\empty}}%
{\ensuremath{\sigma}\xspace}%
{\mbox{\ensuremath{\sigma=\Scm{#1}}}}%
}
% \newcommand{\cond} [1][\empty]%
% {%
% \ifthenelse{\equal{#1}{\empty}}%
% {\c}%
% {\c[#1]}%
% }

\newcommand{\CLong} [0]{doping concentration\xspace}
\newcommand{\CLongL} [0]{doping concentration \ensuremath{C}\xspace}
\newcommand{\CLongs} [0]{doping concentrations\xspace}
\let\Corg\C % Backup
\renewcommand{\C} [1][\empty]%
{%
\ifthenelse{\equal{#1}{\empty}}%
{\ensuremath{C}\xspace}%
{\mbox{\ensuremath{C=\mr{#1}}}}%
}
\newcommand{\Ckl} [1]{\mbox{\ensuremath{C<\mr{#1}}}}
\newcommand{\Cgr} [1]{\mbox{\ensuremath{C>\mr{#1}}}}

% === WORDS ===
\newcommand{\SC}  [0] {semiconductor\xspace}
\newcommand{\SCs} [0] {semiconductors\xspace}
\newcommand{\CSC} [0] {conventional semiconductor\xspace}
\newcommand{\CSCs}[0] {conventional semiconductors\xspace}
\newcommand{\OSC} [0] {organic semiconductor\xspace}
\newcommand{\OSCs}[0] {organic semiconductors\xspace}
\newcommand{\SMU} [0] {source measure unit\xspace}

\newcommand{\IE}  [0] {ionization energy\xspace}
\newcommand{\EA}  [0] {electron affinity\xspace}
\newcommand{\IEs}  [0] {ionization energies\xspace}
\newcommand{\EAs}  [0] {electron affinities\xspace}
\newcommand{\HOMO}[0] {highest occupied molecular orbital\xspace}
\newcommand{\LUMO}[0] {lowest unoccupied molecular orbital\xspace}

\newcommand{\insitu}[0] {in-situ\xspace}
\newcommand{\etal}[0] { et~al.\xspace}
% \xspace oder '\ ' weil sonst zuviel Leer  nach Punkt
\newcommand{\eg}[0]  {\mbox{e.g.}\xspace} % lat: exempli gratia: "zum Beispiel → z.B." (for example)
\newcommand{\ie}[0]  {\mbox{i.e.}\xspace} % lat: id est: "das heißt → d.h." (that is)
\renewcommand{\etc}[0] {\mbox{etc.}\xspace} % lat: et cetera, "und so" (and so on)
\newcommand{\vs}[0] {vs.\ }

% \newcommand{\linesGuides}[0] {Lines are guides to the eye.\xspace}

\newcommand{\mphOFET}[0]{\footnote{measured by Moritz Philipp Hein (IAPP) in an OFET geometry on SiO$_2$ substrate}\xspace}

% WIKI: in der Physik bezeichnet In-situ-Probenpräparation, dass eine Probe unter Ultrahochvakuumbedingungen hergestellt wurde und auch sofort gemessen wird, ohne dass sie das Vakuum verlässt

% === VARIABLES ===
\newcommand{\fFD}     [0] {\ensuremath{f_\text{FD}}\xspace}
\newcommand{\fFDLong} [0] {Fermi-Dirac distribution function\xspace}
\newcommand{\fB}      [0] {\ensuremath{f_\text{B}}\xspace}
\newcommand{\fBLong}  [0] {Boltzmann distribution function\xspace}
\newcommand{\dos}     [0] {\ensuremath{DOS}\xspace}
\newcommand{\dosLong} [0] {density of states\xspace}
\newcommand{\dosLongL} [0] {density of states \ensuremath{DOS}\xspace}
\newcommand{\dosLongs} [0] {densities of states\xspace}
\newcommand{\dosLongsL} [0] {densities of states \ensuremath{DOS}\xspace}

\newcommand{\Td}  [0] {\ensuremath{T_\text{d}}\xspace}
\newcommand{\n} [0] {\ensuremath{n}\xspace}
\newcommand{\nLong} [0] {density of free charge carriers\xspace}
\let\notequal\ne % \notequal -> != /usually \ne, but I need \ne for n_e
\renewcommand{\ne} [0] {\ensuremath{n_\text{e}}\xspace}
\newcommand{\neLL} [0] {\ensuremath{n_\text{e,LL}}\xspace}
\newcommand{\neLong} [0] {density of free electrons\xspace}
\newcommand{\neLongL} [0] {\neLong \ensuremath{n_\text{e}}\xspace}
\newcommand{\neLongs} [0] {densities of free electrons\xspace}
\newcommand{\neLongsL} [0] {\neLongs \ensuremath{n_\text{e}}\xspace}
\newcommand{\nh} [0] {\ensuremath{n_\text{h}}\xspace}
\newcommand{\nhLL} [0] {\ensuremath{n_\text{h,LL}}\xspace}
\newcommand{\nhLong} [0] {density of free holes\xspace}
\newcommand{\nhLongL} [0] {\nhLong \ensuremath{n_\text{h}}\xspace}
\newcommand{\nhLongs} [0] {densities of free holes\xspace}
\newcommand{\nhLongsL} [0] {\nhLongs \ensuremath{n_\text{h}}\xspace}
\newcommand{\neh} [0] {\ensuremath{n_\text{e/h}}\xspace}
\newcommand{\nehLL} [0] {\ensuremath{n_\text{e/h,LL}}\xspace}

\newcommand{\nM} [0] {\ensuremath{n_\text{Mol}}\xspace}
\newcommand{\nMLong} [0] {total density of molecules\xspace}
\newcommand{\nH} [0] {\ensuremath{n_\text{H}}\xspace}
\newcommand{\nHLong} [0] {density of host molecules\xspace}
\newcommand{\nD} [0] {\ensuremath{n_\text{D}}\xspace}
\newcommand{\nDLong} [0] {density of dopant molecules\xspace}
\newcommand{\nDLongL} [0] {density of dopant molecules \ensuremath{n_\text{D}}\xspace}

\newcommand{\DopEff} [1][\empty]{\ifthenelse{\equal{#1}{\empty}}%
 {\ensuremath{\eta_\text{dop}}\xspace}%
 {\mbox{\ensuremath{\eta_\text{dop}=\SI{#1}{\percent}}}}%
}
\newcommand{\DopEffLL} [1][\empty]{\ifthenelse{\equal{#1}{\empty}}%
 {\ensuremath{\eta_\text{dop,LL}}\xspace}%
 {\mbox{\ensuremath{\eta_\text{dop,LL}=\SI{#1}{\percent}}}}%
}
\newcommand{\DopEffLong} [0] {doping efficiency\xspace}
\newcommand{\DopEffLongs} [0] {doping efficiencies\xspace}
\newcommand{\DopEffLongL} [0] {\DopEffLong \ensuremath{\eta_\text{dop}}\xspace}

\let\weg\ni % was some math symbol
\renewcommand{\ni}[0]{\ensuremath{n_\text{i}}\xspace}

\newcommand{\EvLong} [0] {valence energy\xspace}
\newcommand{\Ev} [0] {\ensuremath{E_\text{V}}\xspace}
\newcommand{\EcLong} [0] {conduction energy\xspace}
\newcommand{\Ec} [0] {\ensuremath{E_\text{C}}\xspace}
\newcommand{\Egap} [1][\empty]{\ifthenelse{\equal{#1}{\empty}}%
 {\ensuremath{E_\text{gap}}\xspace}%
 {\mbox{\ensuremath{E_\text{gap}=\eV{#1}}}}% eV!!!
}
\newcommand{\Ed} [0] {\ensuremath{E_\text{D}}\xspace}
\newcommand{\Ea} [0] {\ensuremath{E_\text{A}}\xspace}
\newcommand{\EtLong} [0] {transport level\xspace}
\newcommand{\EtLongL} [0] {\EtLong \ensuremath{E_\text{Tr}}\xspace}
\newcommand{\Et} [1][\empty]{\ifthenelse{\equal{#1}{\empty}}%
 {\ensuremath{E_\text{Tr}}\xspace}%
 {\mbox{\ensuremath{E_\text{Tr}=\meV{#1}}}}%
}

\newcommand{\EfLong} [0] {Fermi level\xspace}
\newcommand{\EfLongL} [0] {\EfLong \ensuremath{E_\text{F}}\xspace}
\newcommand{\Ef} [0] {\ensuremath{E_\text{F}}\xspace}

\newcommand{\EsLong} [0] {Seebeck energy\xspace}
\newcommand{\EsLongL} [0] {\EsLong \ensuremath{E_\text{S}}\xspace}
\newcommand{\Es} [1][\empty]{\ifthenelse{\equal{#1}{\empty}}%
 {\ensuremath{E_\text{S}}\xspace}%
 {\mbox{\ensuremath{E_\text{S}=\meV{#1}}}}%
}

\newcommand{\EactLong} [0]{activation energy of the conductivity\xspace}
\newcommand{\EactLongL} [0]{\EactLong \ensuremath{E_{\text{act,}\c}}\xspace}
\newcommand{\EactLongs} [0]{activation energies of the conductivity\xspace}
\newcommand{\Eact} [1][\empty]{\ifthenelse{\equal{#1}{\empty}}%
 {\ensuremath{E_{\text{act,}\c}}\xspace}%
 {\mbox{\ensuremath{E_{\text{act,}\c}=\meV{#1}}}}%
}

% \newcommand{\Eoffset} [0] {\ensuremath{E_{\text{offset}}}\xspace}

\newcommand{\Nv} [0] {\ensuremath{N_\text{V}}\xspace}
\newcommand{\Nc} [0] {\ensuremath{N_\text{C}}\xspace}
\newcommand{\Nd} [0] {\ensuremath{N_\text{D}}\xspace}
\newcommand{\Ndi}[0] {\ensuremath{N_\text{D}^+}\xspace}
\newcommand{\Na} [0] {\ensuremath{N_\text{A}}\xspace}
\newcommand{\Nai}[0] {\ensuremath{N_\text{A}^-}\xspace}
\newcommand{\gd} [0] {\ensuremath{g_\text{D}}\xspace}
\newcommand{\ga} [0] {\ensuremath{g_\text{A}}\xspace}

\newcommand{\kB} [0] {\ensuremath{k_\text{B}}\xspace}
\newcommand{\kT} [0] {\ensuremath{k_\text{B}T}\xspace}

\newcommand{\EField} [0] {\ensuremath{\mathcal{E}}\xspace}

\newcommand{\dc} [0] {\ensuremath{d_\text{c}}\xspace} % contact distance
\newcommand{\lc} [0] {\ensuremath{l_\text{c}}\xspace} % contact length
\newcommand{\hl} [0] {\ensuremath{h_\text{l}}\xspace} % layer height / thickness

%\newcommand{\mob}  [0] {\ensuremath{\mu}\xspace}
\newcommand{\mob}  [1][\empty]{\ifthenelse{\equal{#1}{\empty}}%
 {\ensuremath{\mu}\xspace}%
 {\mbox{\ensuremath{\mu=\cmVs{#1}}}}%
}
\newcommand{\mobLL}  [1][\empty]{\ifthenelse{\equal{#1}{\empty}}%
 {\ensuremath{\mu_\text{LL}}\xspace}%
 {\mbox{\ensuremath{\mu_\text{LL}=\cmVs{#1}}}}%
}
\newcommand{\mobUL}  [1][\empty]{\ifthenelse{\equal{#1}{\empty}}%
 {\ensuremath{\mu_\text{UL}}\xspace}%
 {\mbox{\ensuremath{\mu_\text{UL}=\cmVs{#1}}}}%
}
\newcommand{\mobe} [0] {\ensuremath{\mob_\text{e}}\xspace}
\newcommand{\mobh} [0] {\ensuremath{\mob_\text{h}}\xspace}
\newcommand{\mobeh} [0] {\ensuremath{\mob_\text{e/h}}\xspace}

\newcommand{\Vs} [0] {\ensuremath{V_\text{S}}\xspace} % Seebeck Voltage
\let\Sorg\S % \S was §
\newcommand{\SLong}[0]{Seebeck coefficient\xspace}
\newcommand{\SLongL}[0]{\SLong \ensuremath{S}\xspace}
\newcommand{\SLongs}[0]{Seebeck coefficients\xspace}
\renewcommand{\S} [1][\empty]{\ifthenelse{\equal{#1}{\empty}}%
 {\ensuremath{S}\xspace}%
 {\mbox{\ensuremath{S=\uVK{#1}}}}%
}
%\newcommand{\MolDensity}  [0] {\ensuremath{\rho_\text{Mol}}\xspace}
% \MolDensity -> \nM
\newcommand{\MM}  [0] {\ensuremath{M}\xspace} % Molar Mass
\newcommand{\density}  [0] {\ensuremath{\rho}\xspace}

\newcommand{\gausswidth} [1][\empty]{\ifthenelse{\equal{#1}{\empty}}%
 {\ensuremath{\sigma_\text{G}}\xspace}%
 {\mbox{\ensuremath{\sigma_\text{G}=\meV{#1}}}}%
}

\newcommand{\gausscenter} [0] {\ensuremath{E_\text{G}}\xspace}
\newcommand{\peltier}     [0] {\ensuremath{\Pi}\xspace}
\newcommand{\avogadro}    [0] {\ensuremath{N_\text{Avo}}\xspace}
\newcommand{\rms}         [0] {\ensuremath{R_\text{rms}}\xspace}
\newcommand{\pKa}         [0] {pKa\xspace}
\newcommand{\druck}       [0] {\ensuremath{P}\xspace} % in Bildern verwendet, z.B. killing+reanimating-C60-2-evap

%\newcommand{\MR} [0] {\mr{}\xspace}

% === MATERIALS ===
% p
\newcommand{\meo} [0] {\mbox{MeO-TPD}\xspace}
\newcommand{\meoLong} [0] {N,N,N',N'-tetrakis 4-methoxyphenyl-benzidine\xspace}
\newcommand{\lili} [0] {\mbox{BF-DPB}\xspace}
\newcommand{\liliLong} [0] {N,N'-Bis(9,9-dimethyl-fluoren-2-yl)-N,N'-diphenyl-benzidine\xspace}
\newcommand{\pen} [0] {pentacene\xspace}
\newcommand{\bphenLong}[0]{4,7-diphenyl-1,10-phenanthroline\xspace}
\newcommand{\znpc} [0] {ZnPc\xspace}
\newcommand{\CSF} [0] {\texorpdfstring{C$_{60}$F$_{36}$}{C60F36}\xspace}
\newcommand{\FS} [0] {\texorpdfstring{\mbox{F$_{6}$-TCNNQ}}{F6-TCNNQ}\xspace}
\newcommand{\FSLong}[0] {1,3,4,5,7,8-hexafluorotetracyanonaphthoquinodimethane\xspace}
\newcommand{\FV} [0] {\texorpdfstring{\mbox{F$_{4}$-TCNQ}}{F4-TCNQ}\xspace}
\newcommand{\FVLong}[0] {tetrafluoro-tetracyanoquinodimethane\xspace}
% n
\newcommand{\CS} [0] {\texorpdfstring{C$_{60}$}{C60}\xspace}
\newcommand{\CrPd}[0] {\texorpdfstring{Cr$_2$(hpp)$_4$}{Cr2(hpp)4)}\xspace}
\newcommand{\WPd}[0] {\texorpdfstring{W$_2$(hpp)$_4$}{W2(hpp)4)}\xspace}
\newcommand{\CrPdLong}[0] {tetrakis(1,3,4,6,7,8-hexahydro-2H-pyrimido[1,2-a]pyrimidinato)\-dichromium (II)\xspace}
\newcommand{\WPdLong}[0]  {tetrakis(1,3,4,6,7,8-hexahydro-2H-pyrimido[1,2-a]pyrimidinato)\-ditungsten (II)\xspace}
\newcommand{\aob} [0] {AOB\xspace}
\newcommand{\aobLong} [0] {3,6-bis(dimethylamino)acridine\xspace}
\newcommand{\dmbiPOH}[0] {\mbox{DMBI-POH}\xspace}
\newcommand{\dmbi}[0] {\dmbiPOH}
\newcommand{\dmbiPOHLong}[0] {2-(1,3-dimethyl-1\textsl{H}-benzoimidazol-3-ium-2-yl)phenolatehydrate\xspace}
\newcommand{\OHdmbi}[0] {\mbox{OH-DMBI}\xspace}
\newcommand{\OHdmbiLong}[0]  {2-(1,3-dimethyl-2,3-dihydro-1\textsl{H}-benzoimidazol-2-yl)phenol\xspace}
\newcommand{\meodmbiI}[0] {\mbox{\textsl{o}-MeO-DMBI-I}\xspace}
\newcommand{\meodmbiILong}[0] {2-(2-methoxyphenyl)-1,3-dimethyl-1\textsl{H}-benzoimidazol-3-ium iodide\xspace}
\newcommand{\meodmbi}[0] {\mbox{\textsl{o}-MeO-DMBI}\xspace}
\newcommand{\meodmbiLong}[0] {2-(2-methoxyphenyl)-1,3-dimethyl-2,3-dihydro-1\textsl{H}-benzoimidazol\xspace}
\newcommand{\Ndmbi}[0] {\mbox{N-DMBI}\xspace}
\newcommand{\NdmbiLong}[0]  {(4-(1,3-dimethyl-2,3-dihydro-1\textsl{H}-benzoimidazol-2-yl)phenyl)dimethylamine\xspace}
\newcommand{\PCBM}[0]  {PCBM\xspace}
\newcommand{\PCBMLong}[0]  {[6,6]-phenyl-C$_{61}$-butyric acid methyl ester\xspace}

\newcommand{\cBild}[4][tb]{
\begin{figure}[#1]% not h at first place!
\centering%
\includegraphics{plot/#2.pdf}%  the '%' is important!!! , removed width [width=#3]
\setcapwidth[c]{\tmCapWidth}%
\caption[#3]{#4}% \caption[shortcaption]{caption}
\label{fig:#2}%
\end{figure}%
}

\newcommand{\cBildDraw}[4][tb]{
\begin{figure}[#1] % not h at first place!
\centering
\includegraphics{draw/#2.pdf}%  the '%' is important!!! , removed width [width=#3]
\setcapwidth[c]{\tmCapWidth}%
\caption[#3]{#4}% \caption[shortcaption]{caption}
\label{fig:#2}
\end{figure}
}

% === Special Commands ===
% in equations
\newcommand{\PUNKT}[0]{\ensuremath{~.}}
\newcommand{\KOMMA}[0]{\ensuremath{~,}}

\newcommand{\sub}  [1] {\ensuremath{_\text{#1}}}

% % Uppercase
% \let\Uorg\U
% \renewcommand{\U}[1]{% Thanks to steven.b.segletes.civ@mail.mil
% \encodetoken\xspace%
% \capitalize[e]{#1}\retokenize[v]{\thestring}%
% \decodetoken\xspace%
% }

% === Refs ===
\newcommand{\Eqnref}[1] {Equation~(\ref{eq:#1})}
\newcommand{\eqnref}[1] {equation~(\ref{eq:#1})}
\newcommand{\eqnrefPage}[1] {\eqnref{#1} on page~\pageref{eq:#1}}
\renewcommand{\eqref}[1] {(\ref{eq:#1})}
\newcommand{\Figref}[1] {Figure~\ref{fig:#1}}
\newcommand{\figref}[1] {figure~\ref{fig:#1}}
\newcommand{\figrefPage}[1] {figure~\ref{fig:#1} on page~\pageref{fig:#1}}
%\newcommand{\figref}[1] {Fig.~\ref{fig:#1}}
\newcommand{\secref}[1] {section~\ref{sec:#1}}
\newcommand{\Secref}[1] {Section~\ref{sec:#1}}
% \newcommand{\secrefPage}[1] {section~\ref{sec:#1} on page \pageref{sec:#1}}
\newcommand{\tabref}[1] {table~\ref{tab:#1}}
\newcommand{\tabrefPage}[1] {table~\ref{tab:#1} on page~\pageref{tab:#1}}
\newcommand{\todosecref}[0] {section~\textcolor{red}X}

% % print the number of the last equation
% \lasteqn[n] -> (n+1)th last; (n = optional)
\newcommand{\lasteq}[1][0]{%
(%
\arabic{chapter}%
.%
%
\addtocounter{equation}{-#1}\arabic{equation}\addtocounter{equation}{#1}%
)\xspace%
}
\newcommand{\lasteqn}[1][0]{% same + word equation
equation~(%
\arabic{chapter}%
.%
%
\addtocounter{equation}{-#1}\arabic{equation}\addtocounter{equation}{#1}%
)\xspace%
}

% a counter I use temporary at some points
\newcounter{tempCounter}
\let\theequationBackup\theequation

% Alternative to \cleardoublepage, but for 'left' pages
\newcommand*\cleartoleftpage{%
  \clearpage
  \ifodd\value{page}\hbox{}\thispagestyle{empty}\newpage\fi
}

% % print the number of the next equation
% % \nexteqn[n] -> last + n (n = optional)
% \newcommand{\nexteqn}[1][1]{
% (\arabic{section}.\addtocounter{equation}{#1}\arabic{equation}\addtocounter{equation}{-#1})
% }

%\renewcommand{\was}{}
%\renewcommand{\name}[1]{#1}
%\renewcommand{\word}[1]{#1}
