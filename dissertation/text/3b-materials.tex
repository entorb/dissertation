\newpage
\section{Materials}\label{sec:Mat}
In this section, the organic materials investigated during this thesis are briefly introduced. Their structures are presented in \figref{mat} and their key properties are summarized in \tabrefPage{MatProp}. A conversion between the \CLong expressed in molar ratio (\mr{}) and in weight\% (\wt{}) for typical values is given in \tabrefPage{Dotierumrechnungen}. Most materials were purified by vacuum sublimation prior to sample fabrication.

\cBildDraw[p]{mat}
{Molecular structures}
{(a)-(c) Structures of the host and dopant compounds investigated in this work. (d) 3-dimensional structure of \CrPd and \WPd, reproduced after\cite{Cotton2003}. (e1) Proposed reaction for formation of the neutral dopant compound \meodmbi from the air-stable precursor \meodmbiI. (e2) Resonant structure of \dmbi and proposed reaction for formation of the neutral dopant \OHdmbi.
}

\subsection{Host Materials}
\label{sec:MatHosts}\label{sec:MatC60}\label{sec:MatMeO}

\minisec{Fullerene \boldmath \CS}
%\was{\CS}
Fullerenes are spherical or ellipsoid molecules containing only carbon atoms, similar to graphene. They are named after the American architect
\name{Richard Buckminster Fuller}, known for his geodesic domes.
%
First theoretical predictions of this class of molecules have been published in 1970 by Osawa\cite{Osawa1970,Osawa1993}, 15 years before the first successful synthesis of the approximately spherical \CS molecule\cite{Kroto1985} which has been awarded with the Nobel Prize in chemistry in 1996. For further information about the electrical and optical properties of fullerenes, the reader is referred to reference\cite{Makarova2001}.

\CS, the fullerene consisting of sixty spherically aligned carbon atoms is a semiconductor with a remarkably high electron mobility of up to \mob[4.9]\cite{Itaka2006}\footnote{measured in an organic field-effect transistor (OFET) geometry on top of a \pen monolayer}. Therefore, it is commonly used in electron transport layers of organic photovoltaic cells (OPV) and hence an interesting subject for studying n-doped layers. \CS molecules with their spherical shape and diameter of approximately \SI{7}{\angstrom} preferably align in a face-centered cubic (fcc) polycrystalline structure\cite{Peimo1993}. Air-exposure of vacuum deposited undoped or doped \CS layers has been reported to result in a decrease of conductivity, attributed to oxygen absorption\cite{Hamed1993,Fujimori1994}. A much smaller decrease has been observed for exposure to N$_2$\cite{Fujimori1994}.
%The key properties of \CS are summarized in \tabrefPage{MatProp}.
Its \EA EA of \eV{4.0(3)}\cite{Zhao2009} requires n-dopants with an \IE IE in a similar range or even below, to allow for electron transfer and doping.
% \eV{4.0(1)}\cite{Yoshida2012}
Interestingly, it was shown that \CS molecules in the solid state are rotating rapidly and nearly isotropically\cite{Tycko1991}. Recently, the presence of fullerenes has been detected in outer space\cite{Cami2010,Lohrmann1823}.
The \CS used in this work has been purchased from CreaPhys GmbH, Germany, purified by vacuum gradient sublimation and was used as received.

\minisec{\meo} %\label{sec:MatMeO}
\meo (\meoLong) has been first synthesized in 1998\cite{Thelakkat1998} as a novel hole transport material for organic light-emitting diodes (OLEDs). It forms amorphous layers\cite{Pfeiffer2003} with hole mobilities in the range of \mob[2.3E-05]\mphOFET. \meo has a low \IE of \mbox{$\text{IE}=\eV{5.10(13)}$}\cite{Olthof2009,Tietze2012} and a wide bandgap in the range of \Egap[3.2]\cite{He2004a}.
Theoretical studies found the second highest occupied molecular orbital (HOMO-1) to be only \eV{0.4} below the HOMO\cite{Matis2010}.
The material has been successfully used in OLEDs\cite{He2004} and organic photovoltaic cells (OPV)\cite{Drechsel2004}, but due to a rather low glass transition temperature of only \Tg[67]\cite{Thelakkat1998}, devices incorporating this material cannot withstand elevated temperatures. Nevertheless, in this thesis it is chosen as one of the host materials for p-doped layers, due to its relevance for research and its rather low \IE. 
\meo has been bought from Sensient Technologies Corporation, USA, and was purified twice by vacuum gradient sublimation.

\minisec{\lili}
\lili (\liliLong) is the second hole transport material investigated in this thesis. This material has been first presented in 2001\cite{Hashimoto2001} for the use in OLED hole transport layers.
\lili has the same backbone as \meo, but different side groups that lead to a \eV{0.13} larger \IE of \mbox{$\text{IE}=\eV{5.23(13)}$}\cite{Meerheim2011}.
Due to the structural similarity to \meo, \lili is expected to form amorphous layers as well. Tests showed that \lili is thermally more stable than \meo. Hole mobilities in the range of \mob[5.7E-05]\mphOFET have been measured, which are twice as high as for \meo.
\lili has been bought from Sensient Technologies Corporation, USA, and was purified by vacuum gradient sublimation.

\minisec{Pentacene}
Pentacene is a planar polycyclic aromatic hydrocarbon molecule consisting of five linearly fused benzene rings, leading to a high conjugation and semiconductor properties. This molecule has a length in the range of \SI{14}{\angstrom}, twice as long as the diameter of \CS molecules. The first synthesis has been described over 100 years ago\cite{Mills1912}.
Pentacene has been one of the first conjugated organic oligomers used as p-type \SC and is still used as reference for all newly developed organic semiconductors\cite{Murphy2007}. The crystalline structure\cite{Ha2009} with rather large crystallite size\cite{Kleemann2012a} leads to high carrier mobilities above \cmVs{1}\cite{Murphy2007}\footnote{measured in an OFET geometry}. Pentacene is prominent for application in organic field-effect transistors (OFETs), as complementary to \CS with a similar mobility, \pen is stable to air-exposure, allowing for lithographic processing steps\cite{Steudel2006}. Due to degradation under UV light, \pen cannot be used in photovoltaic cells.
Pentacene has a low \IE in the range of \mbox{$\text{IE}=\eV{4.90}$}\cite{Salzmann2012} to \eV{5.15}\cite{Fukagawa2006}.
In 2009 high resolution atomic force microscopy (AFM) investigations of single molecules of \pen on Cu(111) have been presented\cite{Gross2009}, impressively resolving the atomic positions and bonds.
Pentacene has been bought from Sensient Technologies Corporation, USA,
and was purified three times by vacuum gradient sublimation.

\subsection{n-Dopants}\label{sec:Matn}
%
\minisec
{\boldmath \CrPd and \WPd}
The n-dopants \CrPd and \WPd are dimetal complexes of chromium or tungsten with the anion of 1,3,4,6,7,8-hexahydro-2H-pyrimido[1,2-a]pyrimidine (hpp). These compounds were first presented by Cotton\etal\cite{Cotton2002,Cotton2005} in 2002, and their use in OLEDs has been reported\cite{PatentWerner2005, WellmannDiss}.
%
Both materials exhibit an extremely low \IE (IE). Using ultraviolet photoelectron spectroscopy (UPS) Selina Olthof (IAPP) measured\footnote{UPS spectra measured using a Specs Phoibos 100 setup at a base pressure below \pa{E-8}. Details about the setup can be found in reference\cite{SelinaOlthofDiss}.}
the IEs of pure \CrPd and \WPd films to \eV{3.95(13)} and \eV{2.68(13)}, respectively\cite{Menke2012}. The IE of \WPd is even shallower than for cesium (\mbox{$\text{IE}=\eV{3.9}$}), the least electronegative stable element. % Francium has smaller IE, but is radioactive
As these compounds easily oxidize in air, they have to be handled in an inert gas atmosphere or vacuum.
Both dopants have been purchased from Novaled AG, Germany, and were used as received.
% As an outcome of this thesis, a comprehensive study on \CS doped by these dopants was published in 2012.\cite{Menke2012}

\minisec{\aob}
Acridine orange base (\aob, \aobLong) is a fluorescent cationic dye used frequently in biology to distinguish DNA and RNA and to detect microbial content of soil and water.
Its application as air-stable n-dopant for \CS has been first presented in 2006\cite{Li2006}, where its doping mechanism has been explained as follows: During vacuum co-deposition of \aob and \CS, a dyad of a positively charged acridine dye and a \CS anion radical, connected by a C--N chemical bond, is formed and acts as dopant. It has been shown that this process can be accelerated by illumination during deposition.
Today, \aob is known for being diffusive and to contaminate vacuum chambers, therefore, it is hardly used in devices any more.
%
\aob has been purchased from Sigma-Aldrich Co. LLC, Germany, and was purified twice by vacuum gradient sublimation.
% A comparison between \aob and the novel dopant \dmbi, see below, was published as a result of this thesis\cite{Menke2012a}.

\minisec{DMBI Derivatives: \dmbiPOH and \meodmbiI}
The investigation of DMBI (1\textsl{H}-benzoimidazole) derivatives for the use as n-dopants was started by Peng Wei and Benjamin D. Naab from Professor Zhenan Bao's group at Stanford University, USA. They first published the application of \Ndmbi\footnote{\Ndmbi is \NdmbiLong} to dope \PCBM\footnote{\PCBM is \PCBMLong} via solution processing\cite{Wei2010}. During this thesis, in a cooperation with Professor Bao's group \dmbiPOH (\dmbiPOHLong) and \meodmbiI (\meodmbiILong) have been investigated for the use as air-stable n-dopants for vacuum co-deposition.
% This led to two publications\cite{Wei2012,Menke2012a}.

Studies of \meodmbiI by various techniques suggest that during co-deposition, a reduction reaction takes place that generates the active doping compound, likely \meodmbi, as shown in \figref{mat}~(e1).
It seems evident that heating leads to the formation of its neutral compound as the active dopant, together with the concomitant loss of the iodide ion.
Hence, for calculation of the molar \CLong of samples doped by \meodmbiI, the molar mass of \meodmbi has to be used, compare \tabrefPage{MatProp}.
Details of the doping mechanism are presented in \secref{ResMeO-DMBI}. %These findings have been published 2012\cite{Wei2012}.
%
\meodmbiI has been synthesized at Stanford University, USA, and was used as received.

\dmbiPOH has been purchased as \OHdmbi (\OHdmbiLong) from Sigma-Aldrich Co. LLC., USA and was used as received. A conversion under ambient conditions to a resonant structure was detected by the cooperation partners at Stanford University. The resonant structure was characterized as shown in \figref{mat}~(e2) by $^1$H-NMR and elemental analysis. By examining the acid dissociation constants (\pKa) of these two resonant structures, it is evident that the equilibrium should favor the phenolate hydrate structure of \dmbiPOH.

Similar to \meodmbiI, \dmbiPOH is expected to form its neutral compound, \OHdmbi (\OHdmbiLong), as the active n-type dopant. A reduction reaction during co-deposition is possible, similar to the one of \meodmbiI due to their structural similarity, especially with \dmbiPOH's resonant structure with an OH$^-$ anion.
The structure of \OHdmbi is shown in \figref{mat}~(e2).
Therefore, in order to calculate the molar \CLong of samples doped by \dmbiPOH, the molar mass of \OHdmbi has to be used, compare \tabrefPage{MatProp}.

\subsection{p-Dopants}\label{sec:Matp}
Typical molecular p-dopants are strongly electron attracting compounds, usually containing fluorine atoms. In this thesis, three different p-dopants have been used.

\minisec{\FV}
\FV (\FVLong), has been first synthesized by Wheland\etal\cite{Wheland1975} and has been used as p-dopant for many years\cite{Blochwitz1998,Pfeiffer1998}, having an \EA of \mbox{$\text{EA}=\eV{5.25}$}\cite{Gao2001}.
Unfortunately, this light compound has been found to be highly diffusive having a low sticking coefficient\cite{Koech2010}. This led to device degradation and contamination of vacuum chambers.
\FV has been purchased from abcr GmbH \& Co. KG, Germany, and was used as received.

\minisec{\FS}
\FS (\FSLong) is a close relative to the prominent molecular p-dopant \FV, with a second aromatic ring, leading to an increase of molar mass by \sfrac{1}{3} and a \K{20} higher deposition temperature \Tdep, compare \tabref{MatProp}. It has been presented by Koech\etal\cite{Koech2010} to replace the volatile \FV and its \EA has been estimated to \mbox{$\text{EA}=\eV{5.0}$}\cite{Tietze2012} which allows successfully doping of different host materials\cite{Tietze2012,Kleemann2012a}.
\FS has been purchased from Novaled AG, Germany, and was used as received.

\minisec{\CSF}
\CSF, a fluorinated derivative of the fullerene \CS was first synthesized in 1991\cite{Selig1991}. The utilization of this heavy compound as p-dopant has been demonstrated in 2011, where its \EA has been estimated to \mbox{$\text{EA}=\eV{5.38}$}\cite{Meerheim2011}.
Despite its molar mass being almost twice as heavy as \CS and its diameter of around \SI{10.5}{\angstrom} being 50\,\% larger, the deposition temperature of \CSF is \K{235} lower, but above the values for the other two investigated p-dopants.
\CSF has been purchased from MTR Ltd, USA, and was used as received.

\vspace*{1cm}

\begin{table}[h]
\centering
\caption
[Conversion between molar ratio and weight\%]
{Conversion between two typical units for the \CLong: molar ratio (\mr{}) and weight percentage (\wt{}) for used material combinations, using \eqnrefPage{MRtoWT}.
% \todo{Add ZnPc?}.
The lowest controllable doping concentration of this setup is in the range of \wt{0.5}.}
\label{tab:Dotierumrechnungen}
\begin{tabular}{
c%Host
c%Dopant
%S[tabnumalign=center,tabformat=1.4]% MR (0.5wt%)
S[tabnumalign=center,tabformat=1.4]% MR (1wt%)
S[tabnumalign=center,tabformat=1.4]% MR (10wt%)
r% wt (0.01MR)
r% wt (0.1MR)
}
\toprule
 & & \multicolumn{2}{c}{\wt{} to \mr{}} &\multicolumn{2}{c}{\mr{} to \wt{}}\\
Host & Dopant &
%{lowest MR} &
{\wt{1}} & {\wt{10}} & {\mr{0.01}} & {\mr{0.10}}\\
\midrule
\CS
& \CrPd
% & 0.0055
& 0.0111
& 0.1219
& 0.90\%
& 8.35\%
\\
\CS & \WPd
% & 0.0039
& 0.0079
& 0.0870
& 1.26\%
& 11.33\%
\\
\CS & \aob
% & 0.0136
& 0.0274
& 0.3018
& 0.37\%
& 3.55\%
\\
\CS & \OHdmbi
% & 0.0151
& 0.0303
& 0.3332
& 0.33\%
& 3.23\%
\\
\CS & \meodmbi
% & 0.0142
& 0.0286
& 0.3149
& 0.35\%
& 3.41\%
\\
\midrule
\meo & \FS
% & 0.0084
& 0.0170
& 0.1867
& 0.59\%
& 5.62\%
\\
\meo & \CSF
% & 0.0022
& 0.0044
& 0.0482
& 2.26\%
& 18.75\%
\\
\lili & \FS
% & 0.0100
& 0.0201
& 0.2212
& 0.50\%
& 4.78\%
\\
\lili & \CSF
% & 0.0026
& 0.0052
& 0.0570
& 1.91\%
& 16.31\%
\\
\pen & \FV
% & 0.0051
& 0.0102
& 0.1120
& 0.98\%
& 9.03\%
\\
\bottomrule
\end{tabular}
\end{table}

\begin{sidewaystable}%[htbp]
\centering
\setcapwidth{\textwidth}
\caption[Material parameters]{Material parameters. \Tdep is the temperature at which typically a deposition rate was detected. Accuracies for the IE (usually measured by UPS) are in the range of $\pm\eV{0.13}$, whereas for the EA (usually measured by IPES) larger errors of $\pm\eV{0.30}$ are common.
}
\label{tab:MatProp}
\begin{tabular}{
@{}%
c@{\hspace*{1.75ex}}%Name
S[tabnumalign=center,tabformat=4.1]% g/mol
S[tabnumalign=center,tabformat=1.3]% g/cm³
c%Tsub
c@{}%Suplier
c%CAS
l%IE
l@{}%EA
}
\toprule
  {Name}
& {Molar Mass \MM}
& {Density \density}
& {\Tdep}
& Supplier
& CAS Number
& \multicolumn{1}{c}{IE}
& \multicolumn{1}{c}{EA}
\\
% name
& \si{\gram\per\mole}
& \si{\gram\per\centi\meter\cubed}
& \si{\celsius}
&
&
& \multicolumn{1}{c}{eV}
& \multicolumn{1}{c}{eV}
\\
\midrule
\multicolumn{2}{c}{Host Materials}\\
\midrule
\CS
 & 720.6
 & 1.630
 & 410
 & CreaPhys
 & 99685-96-8
 & 6.35\cite{Zhao2009}$-$6.45\cite{Akaike2008}
 & 3.9\cite{Olthof2012}$-$4.5\cite{Akaike2008}
\\
\meo
 & 608.7
 & 1.463
 & 170
 & Sensient
 & 122738-21-0
 & \eV{5.10}\cite{Olthof2009,Tietze2012}
 & \eV{1.9}\cite{He2004a} % EA
\\
\lili
 & 720.9
 & 1.210
 & 200
 & Sensient
 & 361486-60-4
 & \eV{5.23}\cite{Meerheim2011}
 & % EA
\\
\pen
 & 278.4
 & 1.320
 & 160
 & Sensient
 & 135-48-8
 & 4.90\cite{Salzmann2012}$-$5.15\cite{Fukagawa2006}
 & 2.7\cite{Chan2009}
\\
% \znpc
%  & 577.9
%  & 1.550
%  & 420
%  & CreaPhys
%  & 14320-04-8
%  & 5.07\cite{Widmer2012}% = Max
% $-$5.28\cite{Gao2001}
% \\
\midrule
\multicolumn{2}{c}{n-Dopants}\\
\midrule
\CrPd
 & 656.8
 &
 & 115
 & Novaled
 & 200943-63-1% (scifinder)
 & 3.95\cite{Menke2012}
 & % EA
\\
\WPd
 & 920.4
 &
 & 190
 & Novaled
 & 463931-34-2
 & 2.86\cite{Menke2012}
 & % EA
\\
\aob
 & 265.4
 &
 & 70
 & Sigma-Aldrich
 & 494-38-2
 & % IE
 & % EA
\\
\dmbi
 & 256.3
 &
 & 110
 & Sigma-Aldrich
 & % CAS
 & % IE
 & % EA
\\
\OHdmbi
 & 240.3
 &
 & --
 &
 & 3652-93-5 %(scifinder)
 & % IE
 & % EA
\\
\meodmbiI
 & 380.2
 &
 & 185
 & Stanford Uni.
 & 201416-21-9 %(scifinder)
 & % IE
 & % EA
\\
\meodmbi
 & 254.3
 &
 & --
 &
 & % CAS
 & % IE
 & % EA
\\
\midrule
\multicolumn{2}{c}{p-Dopants}\\
\midrule
\FV
 & 276.1
 &
 & 100
 & abcr
 & 29261-33-4
 & 8.34\cite{Gao2001}
 & 5.25\cite{Gao2001}(IPES)
\\
\FS
 & 362.2
 &
 & 120
 & Novaled
 & 912482-15-6
 &
 & 5.00\cite{Tietze2012}(CV)
\\
\CSF
 & 1404.6
 &
 & 175
 & MTR
 & 1190690-26-6
 & 8.38\cite{Meerheim2011}
 & 5.38\cite{Meerheim2011}(est.)
\\
\bottomrule
\end{tabular}
\setcapwidth[c]{0.9\textwidth}% usefull?
\end{sidewaystable}

